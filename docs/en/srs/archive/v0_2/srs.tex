%\documentclass[srs, twosides]{softproj}
\documentclass{report}
\usepackage[english]{babel} % For a load of tiny changes in the text
\usepackage{url}
%\usepackage{graphicx, ctable, url, hyperref} % graphics, tables, urls and cross-referencing
\setcounter{secnumdepth}{3}
\setcounter{tocdepth}{3}
%\setdocumentname{Software Requirements Specification}
%\setauthorname{Wouter Van Rossem}

\begin{document}
%\begin{projdoc}
\title{Software Requirements Specification}
\author{Wouter Van Rossem}

\maketitle
\begin{center}
	\textbf{Abstract} \\
	This document describes the Software Requirements Specifications for
	the software behind an online auction website. This project is part of 
	the course on software engineering at the Vrije Universiteit Brussel.
	\bigskip
%Table of revisions here
\begin{tabular}{|c|c|c|}
	\hline  \textbf{Revision} & \textbf{Date} & \textbf{Comment} \\ 
	\hline 0.1 & 09/11/2009 & First draft \\ 
	\hline 0.2 & 13/11/2009 & Updated after quality assurance \\ 
	\hline 
\end{tabular} 
\end{center}
\tableofcontents

\chapter{Introduction}
\section{Purpose}

	The purpose of this document is to present an overall description and 
	listing of the functionality of the system behind an online auction 
	website. This document is intended for users of the system including
	designers, testers, implementation unit and the employer.
	
\section{Scope}

	Salesmen is the system behind an online auction website. An auction site 
	is a web application where users can buy and sell objects. Users can 
	place auctions or bid on auctions of other users. 
	
	There are also some social features such as sending messages to users,
	favourite sellers and a rating system for users.
	
\section{Definition And Abbreviation}

	\begin{itemize}
		\item \emph{SRS}: Software Requirements Specification
		\item \emph{UML}: Unified Modelling Language
		\item \emph{XHTML}: Extensible HyperText Markup Language
		\item \emph{HTML}: HyperText Markup Language
		\item \emph{CSS}: Cascading Style Sheets
		\item \emph{HTTPS}: Hypertext Transfer Protocol Secure
		\item \emph{CAPTCHA}: Completely Automated Public Turing test to tell Computers and Humans Apart
		%\item \emph{}:
	\end{itemize}
	
\section{References}
See Appendix A.

\section{Overview}

	Section 2 describes the general factors that affect the product and 
	its requirements. This section does not state specific requirements. Instead, it provides a 
	background for those requirements, which are defined in detail in Section 
	3 of the SRS, and makes them easier to understand.
	
	Section 3 contains all of the software requirements, to a level of detail 
	sufficient to enable designers to design the system to satisfy these 
	requirements, and testers to test that the system satisfies these
	requirements.


\chapter{Overall Description}

\section{Product Perspective}

	The project will make use of a database and a web server that can be 
	accessed with any web browser. There are six types of users on the system:
	guests, unconfirmed users, users, moderators, administrators and banned users. 
	Before a user can make use of the full functionality of the site, the user has to register. The user
	will have to enter some personal information in a form in order to do this. A confirmation mail will then be sent to the unconfirmed user. This mail contains a link that the user has to follow in order to become a confirmed user.
	
	Both guests and users can browse the different auctions, but only registered 
	users can bid on auctions or place their own auctions. Browsing through auctions can be done by categories, tags or searching a specific auction. 
	
	Different auctions can be created such as English auctions\cite{English} or
	silent auctions\cite{silent}. When a user places an auction, he or she also has to specify
	payment methods, transport methods, a minimum price, a duration for the auction
	and other general information about the auction. Some other extra's such as a picture can be added to the auction for a small fee.
	
	A user can follow an auction. The followed auctions list is a list of auctions that interest the user
	but on which he or she may not have bid yet. Then when the user wants, he or she can bid on the auction
	by going to the page of the auction.
	A user can view all the auctions he or she has bid on in the active auctions list.
	
	When a user has won an auction, he or she can pay for the item. This will be done through a 
	transaction. The buyer can pay for the item by choosing on of the payment options the seller 
	has specified. One of the supported methods is through salespal. This a personal ``bank account" on
	the site that each user has. Buyers can pay for items with the money on this account and 
	sellers can receive money on their salespal account. Users can top up their salespal account,
	i.e. they transfer money to the account. When the transaction is done, buyers and sellers 
	can rate the transaction. Each user will have a rating then which is based on the ratings
	of their transactions. 
	
	Users can also contact other users through personal messages. Other sellers can also
	be added to a favourite sellers list, so that a user can easily check if a 
	seller they like has new auctions. Auctions can also be recommended to users on the 
	page of an auction or on the home page. Users also have a buyer's assistant. A user
	can set this up so that he or she can quickly see auctions that he or she is interested
	in.
	
	Auctions also have a popularity meter that depends on the amount of users that view
	the auction page. A list of the ten most popular auctions is also displayed on the
	home page of the site: the hot deals.
	
	Salesman pro is another functionality for which users can pay an extra monthly fee. This
	will give them reductions on the placing fees of auctions, remove advertising on the
	web site, ... Their auctions will also be higher in search results and will go faster 
	on the hot deals list.
	
	Moderators are responsible for managing the users and the auctions. Some functionalities they have are: 
	banning a user, retracting a bid of a user, removing an auction, ... 
	
	Administrators are the users that manage the site. 
	
\section{User Characteristics}
	The users of the system will be users with different levels of 
	technical expertise. Any user with a basic understanding of the 
	internet and auctions should be able to make use of all the 
	available functionality of the system.
	
	There are six different types of users:
	
	\begin{itemize}
		\item \textbf{Guests}: These are visitors of the site which don't 
		have an account yet or aren't logged in.
		\item \textbf{Unconfirmed users}: These are users that haven't confirmed 
		their account by clicking the link in the confirmation mail they have received.
		\item \textbf{Users}: These are users of the site who have confirmed their 
		account and are logged in.
		\item \textbf{Moderators}: These are users that manage the users and the auctions.
		\item \textbf{Administrators}: These are special users of the 
		site who manage the site.
		\item \textbf{Banned users}: These are users that were removed by
		an administrator.
	\end{itemize}
	
\section{Interfaces}

	\begin{itemize}
		\item The auction site is accessible from any operating system using 
			a web browser and a connection to the web server running the
			Salesmen software
		\item No special hardware is required by the end-user
		\item The client browser must be W3C XHTML compatible
		\item Communication between the users and the auction site will be
			through HTTP communication using TCP/IP port 80
		\item If an error occurs during a request, the user should receive 
			a clear error message. These errors should also be logged			
	\end{itemize}
	
	\subsection{User interfaces through forms}
	\textbf{Account information}
		\begin{itemize}
			\item Username (mandatory)
			\item Password (mandatory)
			\item Password verification (mandatory)
			\item E-mail (mandatory)
			\item E-mail verification (mandatory)
			\item Default Language (default English)
			\item CAPTCHA (mandatory)
		\end{itemize}
	\textbf{Personal information}
		\begin{itemize}
			\item First Name (mandatory)
			\item Last Name (mandatory)
			\item Address (optional)
			\item Phone number (optional)
			\item Date of birth (optional)
		\end{itemize}
	\textbf{Registration form}
		\begin{itemize}
			\item Account information (mandatory)
			\item Personal information (mandatory)
		\end{itemize}
	\textbf{Login form}
		\begin{itemize}
			\item Username (mandatory)
			\item Password (mandatory)
		\end{itemize}
	\textbf{Basic search form}
		\begin{itemize}
			\item Web search query (optional)
		\end{itemize}
	\textbf{Advanced search form}
		\begin{itemize}
			\item Include keyword (optional)
			\item Exclude keyword (optional)
			\item Search title (check, optional)
			\item Search description (check, optional)
			\item Category (list, optional)
			\item Tags (optional)
			\item Price range (optional)
			\item Duration range (optional)
			\item Auction type (check, optional)
			\item Payment method (check, optional)
			\item Transport cost range (optional)
			\item Seller name (optional)
		\end{itemize}
	\textbf{Auctions with tag}
		\begin{itemize}
			\item Tag name (optional)
		\end{itemize}
	\textbf{Search user}
		\begin{itemize}
			\item User search query (optional)
		\end{itemize}
	\textbf{Place auction form}
		\begin{itemize}
			\item Auction name (mandatory)
			\item Transport options (mandatory)
			\item Minimum price (mandatory)
			\item Duration (mandatory)
			\item Auction type (mandatory)
			\item Category (mandatory)
			\item Tags (optional)
			\item Picture (optional)
			\item Other information (optional)
		\end{itemize}
	\textbf{Bidding form}
		\begin{itemize}
			\item Maximum bid (mandatory)
		\end{itemize}
	\textbf{Personal message form}
		\begin{itemize}
			\item To (mandatory)
			\item Subject (optional)
			\item Message (mandatory)
		\end{itemize}
	\textbf{Rate transaction form}
		\begin{itemize}
			\item Rate description accuracy (list, mandatory)
			\item Rate shipping (list, mandatory)
			\item Rate contact with seller (list, mandatory)
			\item Message (optional)
		\end{itemize}
	\textbf{Buyer's assistant entry form}
		\begin{itemize}
			\item Advanced search form (mandatory)
		\end{itemize}
	\textbf{Auction comment form}
		\begin{itemize}
			\item Comment message (mandatory)
		\end{itemize}
		
\section{Product Functions}

	\subsection{Guest}
		Guests can
		\begin{itemize}
			\item Browse through auctions by category
			\item Browse through auctions by tags
			\item Browse through auctions by search query
			\item Request an account
			\item Log in
		\end{itemize}
	\subsection{Unconfirmed user}
		Unconfirmed users can
		\begin{itemize}
			\item Confirm account
			\item Request a new confirmation mail
		\end{itemize}
	\subsection{User}
		Users can
		\begin{itemize}
			\item Browse through auctions by category
			\item Browse through auctions by tags
			\item Browse through auctions by search query
			\item Place auctions
			\item Bid on auctions
			\item Follow an auction
			\item View a transaction
			\item Pay for a transaction
			\item Rate a transaction
			\item View their placed, active, won and followed auctions
			\item Access and modify their account information
			\item Send a personal message to another user
			\item View and manage their personal messages
			\item Add a seller to their favourite sellers
			\item View and top up their salespal account
			\item Comment on an auction
			\item Log out
		\end{itemize}
	\subsection{Administrator}
		Administrators can
		\begin{itemize}
			\item Manage the members
			\item Manage the auctions
			\item Retract bids from users
		\end{itemize}
		
\section{Constraints}

\begin{itemize}
	\item The system must work on Linux, and more specifically on Wilma
	\item The design should be modular, so extensions and replacements of 
	modules will be simplified
	\item The web interface should be simple, attractive and standard
	(CSS, XHTML)
	\item The basic programming language must be Java
	\item Only open source software and libraries may be used
\end{itemize}

%\section{Assumptions and Dependencies}

\section{Apportioning of Requirements}

	In order to have a working prototype available at the end of the
	first iteration, some functionalities have different priorities.
	
	\begin{itemize}
		\item \textbf{Must have}: These functionalities should definitely 
		be in the system, preferably after the first iteration.
		\item \textbf{Want to have}: These functionalities should be in 
		the system, but could be dropped when there isn't enough time.
		\item \textbf{Nice to have}: These functionalities should only be 
		implemented if all the `must have' and `want to have' 
		requirements are implemented and there is still some time left.
	\end{itemize}

\chapter{Specific Requirements}

\section{Requirements Control Plan}
	This section describes how the different requirements will be described and how 
	modifications to these requirements will be handled. \\
	Each requirement will have a unique identifier which will be used for referencing 
	in the source code, the design, ... Some requirements will only be described by
	a short description, while others will be described by a use case. \\
	Whenever a requirement changes, the change will be noted at the description of 
	the requirement. If a requirement is retired, this will also be noted at the
	requirement description, together with a reason why the requirement was retired.

%\section{External Interface Requirements}
\section{Functional Requirements} % Organized by user class
% Template
%\subsubsection{Remove an auction}
%			\begin{description}
%				\item[Priority]
%				\item[Actor]
%				\item[Preconditions]
%				\item[Description]
%				\item[Exceptions]
%				\item[Result]
%			\end{description}
	\subsection{Guest}
		\subsubsection{Create a new account} 
			\begin{description}
				\item[Requirement ID] 1
				\item[Priority] Must have
				\item[Actor] Guest
				\item[Preconditions] User is not logged in
				\item[Description]
				A guest can create a new account so that he or she 
				can use the full functionality of the site
 				\item[Main path]
 					\begin{enumerate}
						\item Guest selects \emph{register}
						\item \label{1a} Guest fills in \emph{registration form}
						\item Guest submits form
						\item System checks form and if valid saves it
						\item System sends a confirmation mail 
					\end{enumerate}
				\item[Exceptional paths]
					\begin{enumerate}
						\item[4a] Username is already in use
							\begin{itemize}
								\item[4a1] System informs the user that the username is already taken 
								\item[4a2] User returns to step \ref{1a}, with the correct information still entered in the form field
							\end{itemize}
						\item[4b] E-mail address is already in use
							\begin{itemize}
								\item[4b1] System informs the user that the e-mail address is already in use and asks if the user is sure he or she wants to use this e-mail address
								\item[4b2] The account is created or the user returns to step \ref{1a}, with the correct information still entered in the form field
							\end{itemize}
						\item[4c] Incorrect information in the registration form
							\begin{itemize}
								\item[4c1] System informs the user that there is some incorrect information in the form 
								\item[4c2] User returns to step \ref{1a}, with the correct information still entered in the form field
							\end{itemize}
						\item[4d] Incomplete form
							\begin{itemize}
								\item[4d1] System informs the user that there are some fields in the form that are not filled in
								\item[4d2] User returns to step \ref{1a}, with the correct information still entered in the form field
							\end{itemize}			
					\end{enumerate}
				\item[Result] An account is created for the user
			\end{description}
		\subsubsection{Log in}
			\begin{description}
				\item[Requirement ID] 2
				\item[Priority] Must have
				\item[Actor] Guest
				\item[Preconditions] The user is registered on the site,
				user is not logged in
				\item[Description]
				If a user is registered but not logged in, he or she can
				log in to use the full functionality of the site.
				\item[Main path]
 					\begin{enumerate}
						\item \label{2a} Guest fills in login form
						\item Guest submits form
						\item System checks if the form is valid and if so logs
						the user in
						\item System redirect the user to the page he or she was on
							before the log in
					\end{enumerate}
				\item[Exceptional paths]
					\begin{enumerate}
						\item[3a] Incorrect username and/or password
							\begin{itemize}
								\item[4a1] System informs the user that the he or she has entered an incorrect information
								\item[4a2] User returns to step \ref{2a}
							\end{itemize}
						\item[3b] Incomplete login form
							\begin{itemize}
								\item[4a1] System informs the user that some information is missing
								\item[4a2] User returns to step \ref{2a}
							\end{itemize}
					\end{enumerate}
				\item[Result] The user is logged in
			\end{description}
	\subsection{Unconfirmed user}
		\subsubsection{Confirm account}
			\begin{description}
				\item[Requirement ID] 41
				\item[Priority] Want to have
				\item[Actor] Unconfirmed user
				\item[Preconditions] User is at confirmation page
				\item[Description] Unconfirmed users can confirm their account so they become regular users
				\item[Main path]
 					\begin{enumerate}
						\item Users follows link in the confirmation mail he or she has received
						\item User selects confirm account on the confirmation page
						\item 
					\end{enumerate}
				\item[Exceptions] None
				\item[Result] The user becomes a regular user and can make use of all the functionality of the site
			\end{description}
		\subsubsection{Request new confirmation mail}
			\begin{description}
				\item[Requirement ID] 42
				\item[Priority] Want to have
				\item[Actor] Unconfirmed user
				\item[Preconditions] User is at control panel
				\item[Description] Unconfirmed users can request a new confirmation mail
				\item[Main path]
 					\begin{enumerate}
						\item Users selects "send new confirmation mail"
						\item System send a new confirmation mail to the user's e-mail address
						\item 
					\end{enumerate}
				\item[Exceptions] None
				\item[Result] The user receives a new confirmation mail
			\end{description}
	\subsection{Guest and User}
		\subsubsection{Search for an auction}
			\begin{description}
				\item[Requirement ID] 3
				\item[Priority] Must have
				\item[Actor] User or Guest
				\item[Preconditions] User is at the home page
				\item[Description] Guests and members can search for auctions
				\item[Main path]
 					\begin{enumerate}
						\item Users enters a search term in the search field
						\item User submits the search
						\item System redirects the user to a page with all found auctions that match this search
						query
					\end{enumerate}
				\item[Exceptions] Empty search field
				\item[Result] A page with the found auctions for the search term
			\end{description}
		\subsubsection{View auctions of a certain category}
			\begin{description}
				\item[Requirement ID] 4
				\item[Priority] Must have
				\item[Actor] User or Guest
				\item[Preconditions] User is at the home page
				\item[Description] Guests and users can view all the auctions of a certain category
				\item[Main path]
 					\begin{enumerate}
						\item Users select a category from the category list
						\item System redirects the user to a page with all the auctions of the
							selected category
					\end{enumerate}
				\item[Exceptions] None
				\item[Result] A page displaying all the auctions of a certain category
			\end{description}
		\subsubsection{View auctions with a certain tag}
			\begin{description}
				\item[Requirement ID] 5
				\item[Priority] Want to have
				\item[Description] Guests and users can view all the auctions with a certain tag
			\end{description}
		\subsubsection{Change language}
			\begin{description}
				\item[Requirement ID] 6
				\item[Priority] Must have
				\item[Description] Guests and users can change the language of the website to one
					of the available languages
			\end{description}
		\subsubsection{Change currency}
			\begin{description}
				\item[Requirement ID] 7
				\item[Priority] Nice to have
				\item[Description] Guests and users can change the currency in which auctions are
					displayed
			\end{description}
	\subsection{User}
		\subsubsection{Log out}
			\begin{description}
				\item[Requirement ID] 8
				\item[Priority] Must have
				\item[Actor] User
				\item[Preconditions] User is logged in
				\item[Description] Members who are logged in can log out
				\item[Main path]
 					\begin{enumerate}
						\item User selects \emph{log out}
						\item System logs the user out
						\item System redirects the user to the site's homepage
					\end{enumerate}
				\item[Exceptions] None
				\item[Result] The user is logged out
			\end{description}
		\subsubsection{Search a user}
			\begin{description}
				\item[Requirement ID] 9
				\item[Priority] Want to have
				\item[Actor] User
				\item[Preconditions] User is at advanced search page
				\item[Description] Members can search for other members 
				\item[Main path]
 					\begin{enumerate}
						\item User enters search term in search user field
						\item User submits search
						\item System redirects the user to page containing all users corresponding to that search 
						query
					\end{enumerate}
				\item[Exceptions] Incorrect or incomplete form
				\item[Result] A page displaying the found members for the search term
			\end{description}
		\subsubsection{Place an auction}
			\begin{description}
				\item[Requirement ID] 10
				\item[Priority] Must have
				\item[Actor] User
				\item[Preconditions] User is logged in. User is at home page or user home
				\item[Description] Members of the site can create a new auction on which
				other members can bid
				\item[Main path]
 					\begin{enumerate}
						\item User selects \emph{place auction}
						\item User fills in \emph{new auction form}
						\item User submits the form
						\item System checks the form and if valid creates the auction
					\end{enumerate}
				\item[Exceptions] Incorrect information in the form, incomplete form
				\item[Result] Auction is placed
			\end{description}
		\subsubsection{View placed auctions}
			\begin{description}
				\item[Requirement ID] 11
				\item[Priority] Want to have
				\item[Actor] User
				\item[Preconditions] User is logged in. User is at control panel
				\item[Description] Users can view the auctions they have placed
				\item[Main path]
 					\begin{enumerate}
						\item User selects \emph{placed auctions}
						\item System redirects user to user's placed auctions page
					\end{enumerate}
				\item[Exceptions] None
				\item[Result] The user views the auctions he or she has placed
			\end{description}
		\subsubsection{Bid on an auction}
			\begin{description}
				\item[Requirement ID] 12
				\item[Priority] Must have
				\item[Actor] User
				\item[Preconditions] 
 					\begin{enumerate}
						\item User is logged in
						\item User is on an auction page
						\item Auction is not of the user
					\end{enumerate}
				\item[Description] Members can bid on auctions of other members
				\item[Main path]
 					\begin{enumerate}
						\item User select \emph{bid}
						\item User fills in bidding form
						\item User submits form
						\item System checks form and if valid places the bid
					\end{enumerate}
				\item[Exceptions] Wrong bid value in the form, incomplete form
				\item[Result] The bid is placed on the auction
			\end{description}
		\subsubsection{View active auctions}
			\begin{description}
				\item[Requirement ID] 13
				\item[Priority] Want to have
				\item[Actor] User
				\item[Preconditions] User is logged in. User is at control panel
				\item[Description] Users can view the auctions they have bid on
				\item[Main path]
 					\begin{enumerate}
						\item User selects \emph{active auctions}
						\item System redirects user to user's active auctions page
					\end{enumerate}
				\item[Exceptions] None
				\item[Result] The user can view the auctions he or she had bid on
			\end{description}
		\subsubsection{Modify account information}
			\begin{description}
				\item[Requirement ID] 14
				\item[Priority] Must have
				\item[Actor] User
				\item[Preconditions] User is logged in. User is at control panel
				\item[Description] Members can modify their account information
				\item[Main path]
 					\begin{enumerate}
						\item User selects \emph{modify account}
						\item User changes account information form
						\item User submits the form
						\item System checks the form and if valid, makes the changes
					\end{enumerate}
				\item[Exceptions] Incorrect information, incomplete form
				\item[Result] The account information of the user is changed
			\end{description}
		\subsubsection{Send a personal message}
			\begin{description}
				\item[Requirement ID] 15
				\item[Priority] Nice to have
				\item[Actor] User
				\item[Preconditions] User is logged in. User is at user page
				\item[Description] Users can send messages to other users
				\item[Main path]
 					\begin{enumerate}
						\item User selects \emph{send message}
						\item User fills in personal message form
						\item User submits the form
						\item System checks the form and if valid, sends the message
					\end{enumerate}
				\item[Exceptions] Incorrect information, incomplete form, users send message to him or herself
				\item[Result] A message is sent to another user
			\end{description}
		\subsubsection{View personal messages}
			\begin{description}
				\item[Requirement ID] 16
				\item[Priority] Nice to have
				\item[Actor] User
				\item[Preconditions] User is logged in. User is at control panel
				\item[Description] Users can view the personal messages
				\item[Main path]
 					\begin{enumerate}
						\item User selects \emph{personal messages}
						\item System redirects the user the user's personal messages page
					\end{enumerate}
				\item[Exceptions] None
				\item[Result] The user will view a page with his or her personal messages
			\end{description}
		\subsubsection{Delete a personal message}
			\begin{description}
				\item[Requirement ID] 17
				\item[Priority] Nice to have
				\item[Actor] User
				\item[Preconditions] User is logged in. User is at personal messages page
				\item[Description] Users can delete personal messages they have received
				\item[Main path]
 					\begin{enumerate}
						\item User selects a message
						\item User selects \emph{delete message}
						\item System deletes the message
					\end{enumerate}
				\item[Exceptions] None
				\item[Result] A message is deleted from the user's inbox
			\end{description}
		\subsubsection{Follow an auction}
			\begin{description}
				\item[Requirement ID] 18
				\item[Priority] Want to have
				\item[Actor] User
				\item[Preconditions] User is logged in. User is at auction page
				\item[Description] Users can follow an auction, i.e. they put it in their following list
				\item[Main path]
 					\begin{enumerate}
						\item User selects \emph{follow auction}
						\item System adds auction to the users followed auctions list
					\end{enumerate}
				\item[Exceptions] None
				\item[Result] An auction is added to the user's follow auctions list
			\end{description}
		\subsubsection{View followed auctions}
			\begin{description}
				\item[Requirement ID] 19
				\item[Priority] Want to have
				\item[Actor] User
				\item[Preconditions] User is logged in. User is at control panel
				\item[Description] Users can view the auctions they are following
				\item[Main path]
 					\begin{enumerate}
						\item User selects \emph{followed auctions}
						\item System redirects user to user's followed auctions page
					\end{enumerate}
				\item[Exceptions] None
				\item[Result] The user can view the auctions he or she is following
			\end{description}
		\subsubsection{View Transaction}
			\begin{description}
				\item[Requirement ID] 20
				\item[Priority] Want to have
				\item[Actor] User
				\item[Preconditions] User is logged in. User is at auction page. User bought the item.
				\item[Description] Users can view the transaction of an item they bought
				\item[Main path]
 					\begin{enumerate}
						\item User selects \emph{view transaction}
						\item System forwards user to transaction page of auction
					\end{enumerate}
				\item[Exceptions] None
				\item[Result] The user views the transaction of the auction
			\end{description}
		\subsubsection{Pay transaction}
			\begin{description}
				\item[Requirement ID] 21
				\item[Priority] Want to have
				\item[Actor] User
				\item[Preconditions] User is logged in. User is at transaction page
				\item[Description] Users can pay auctions on the transaction page of an auction
				\item[Main path]
 					\begin{enumerate}
						\item User selects \emph{pay item}
						\item User selects a payment method
						\item User performs the payment
						\item System notifies the seller that the auction is paid for
					\end{enumerate}
				\item[Exceptions] Incorrect information, incomplete form
				\item[Result] The transaction is paid for
			\end{description}
		\subsubsection{Rate transaction}
			\begin{description}
				\item[Requirement ID] 22
				\item[Priority] Want to have
				\item[Actor] User
				\item[Preconditions] User is logged in. User is at transaction page, user paid transaction
				\item[Description] Users can send rate a transaction after it is paid
				\item[Main path]
 					\begin{enumerate}
						\item User selects \emph{rate transaction}
						\item User fills in rate transaction form
						\item User submits the form
						\item System checks the form and if valid, rates the transaction
						\item System updates the ratings of the user
					\end{enumerate}
				\item[Exceptions] Incorrect information, incomplete form
				\item[Result] The transaction is rated
			\end{description}
		\subsubsection{Add seller to favourites}
			\begin{description}
				\item[Requirement ID] 23
				\item[Priority] Nice to have
				\item[Actor] User
				\item[Preconditions] User is logged in. User is at user page
				\item[Description] Users can add a seller to their favourites so they can their auctions easily
				\item[Main path]
 					\begin{enumerate}
						\item User selects \emph{add seller to favourites}
						\item System adds the seller to the favourite seller list of the user
					\end{enumerate}
				\item[Exceptions] Seller is already in the favourite seller list
				\item[Result] The seller is added to the favourite seller list of the user
			\end{description}
		\subsubsection{View favourite sellers}
			\begin{description}
				\item[Requirement ID] 24
				\item[Priority] Nice to have
				\item[Actor] User
				\item[Preconditions] User is logged in. User is at control panel
				\item[Description] Users can view his or her favourite sellers
				\item[Main path]
 					\begin{enumerate}
						\item User selects \emph{favourite sellers}
						\item System redirects user to user's favourite sellers page
					\end{enumerate}
				\item[Exceptions] None
				\item[Result] The user will view a page with his or her favourite sellers
			\end{description}
		\subsubsection{Delete a favourite seller}
			\begin{description}
				\item[Requirement ID] 25
				\item[Priority] Nice to have
				\item[Actor] User
				\item[Preconditions] User is logged in. User is at control panel
				\item[Description] Users can remove a seller from his or her favourite sellers list
				\item[Main path]
 					\begin{enumerate}
						\item User selects a seller from the list
						\item User selects \emph{delete seller} 
						\item System removes the selected user from the list
					\end{enumerate}
				\item[Exceptions] None
				\item[Result] A seller is removed from user's favourite sellers list
			\end{description}
		\subsubsection{View salespal}
			\begin{description}
				\item[Requirement ID] 26
				\item[Priority] Want to have
				\item[Actor] User
				\item[Preconditions] User is logged in. User is at control panel
				\item[Description] Users can view their own personal ``bank acount" on the site
				\item[Main path]
 					\begin{enumerate}
						\item User selects \emph{view salespal}
						\item System redirects the user to the user's salespal page
					\end{enumerate}
				\item[Exceptions] None
				\item[Result] The user is on his or her salespal page
			\end{description}
		\subsubsection{Top up salespal}
			\begin{description}
				\item[Requirement ID] 27
				\item[Priority] Want to have
				\item[Actor] User
				\item[Preconditions] User is logged in. User is at salespal page
				\item[Description] Users can add more money on their salespal account
				\item[Main path]
 					\begin{enumerate}
						\item User selects \emph{top up salespal}
						\item Users fills in top up salespal form
						\item System checks the form and if valid tops up the account
					\end{enumerate}
				\item[Exceptions] Incorrect or incomplete form
				\item[Result] The user tops up his or her salespal account
			\end{description}
		\subsubsection{View recommended auctions}
			\begin{description}
				\item[Requirement ID] 28
				\item[Priority] Nice to have
				\item[Actor] User
				\item[Preconditions] User is logged in. User is at control panel
				\item[Description] Users can view recommended auctions for him or her. This list 
					is generated through tags and categories the user frequently uses
				\item[Main path]
 					\begin{enumerate}
						\item User selects \emph{view recommendations}
						\item System redirects user to user's recommended auctions page
					\end{enumerate}
				\item[Exceptions] None
				\item[Result] The user views a page with recommended auctions for the user
			\end{description}
		\subsubsection{View buyer's assistant}
			\begin{description}
				\item[Requirement ID] 29
				\item[Priority] Want to have
				\item[Description] A user can check the auctions the buyer's assistant has found
					for his or her preferences
			\end{description}
		\subsubsection{Comment on an auction}
			\begin{description}
				\item[Requirement ID] 38
				\item[Priority] Want to have
				\item[Description] A user can comment on an auction. This comment can be viewed by
					anyone viewing the auction page
			\end{description}
	\subsection{Administrator}
		\subsubsection{Remove a user}
			\begin{description}
				\item[Requirement ID] 30
				\item[Priority] Must have
				\item[Actor] Administrator
				\item[Preconditions] User is at control panel
				\item[Description] User selects \emph{manage users} and selects a user from the users list.
				After the user is selected, \emph{remove user} is selected.
				\item[Exceptions] None
				\item[Result] A user is removed from the system
			\end{description}	
		\subsubsection{Remove an auction}
			\begin{description}
				\item[Requirement ID] 31
				\item[Priority] Must have
				\item[Actor] Administrator
				\item[Preconditions] User is at control panel
				\item[Description] User selects \emph{manage auctions} and selects an auction from the auctions 
				list. After the auction is selected, \emph{manage auction} is selected.
				\item[Exceptions] None
				\item[Result] Auction is removed
			\end{description}
		\subsubsection{Retract bid}
			\begin{description}
				\item[Requirement ID] 32
				\item[Priority] Want to have
				\item[Description] An administrator can retract a bid from a user when that user has e.g. 
					made a high bid because of a typo
			\end{description}
	\subsection{Security}
		\subsubsection{Encrypted password}
			\begin{description}
				\item[Requirement ID] 33
				\item[Priority] Must have
				\item[Description] Password must be stored encrypted
			\end{description}
		\subsubsection{CAPTCHA}
			\begin{description}
				\item[Requirement ID] 34
				\item[Priority] Must have
				\item[Description] When a user registers, he or she has to fill in a CAPTCHA
			\end{description}
		\subsubsection{Limited login attempts}
			\begin{description}
				\item[Requirement ID] 35
				\item[Priority] Must have
				\item[Description] A guest may only try to try to log in with a wrong password 
					a fixed number of times
			\end{description}
		\subsubsection{Extra site for banned users}
			\begin{description}
				\item[Requirement ID] 39
				\item[Priority] Nice to have
				\item[Description] When a banned user visits the site, he or she will view a special
					site for banned users
			\end{description}
		\subsubsection{Cookies}
			\begin{description}
				\item[Requirement ID] 40
				\item[Priority] Want to have
				\item[Description] Cookies can be created for the user to let the user log in 
					automatically
			\end{description}
	\subsection{Other}
		\subsubsection{Basic e-mail notification}
			\begin{description}
				\item[Requirement ID] 36
				\item[Priority] Want to have
				\item[Description] User can receive an e-mail when he or she has won an auction,
					when the item is shipped, when an auction is paid for
			\end{description}
		\subsubsection{Advanced e-mail notification}
			\begin{description}
				\item[Requirement ID] 37
				\item[Priority] Nice to have
				\item[Description] User can receive an e-mail when he or she is overbid,
					when an auction is almost done, with recommended auctions for the user
			\end{description}
\section{Non-functional Requirements}
	\subsection{Browser compatibility}
		\begin{description}
			\item[Requirement ID] 39
			\item[Priority] Must have
			\item[Description] The site should be fully compatible with the three most popular browsers. At the time of writing these are: Internet Explorer (version 7 or higher), Firefox (version 3 or higher) and Safari (version 4 or higher)\cite{browsers}.
		\end{description}
	\subsection{Secure connection}
		\begin{description}
			\item[Requirement ID] 40
			\item[Priority] Must have
			\item[Description] Whenever sensitive transactions are made, the HTTPS protocol should be used.
		\end{description}
	\subsection{Advertisments through Google AdWords}
		\begin{description}
			\item[Requirement ID] 43
			\item[Priority] Want to have
			\item[Description] There should be space for advertisements from Google AdWords
		\end{description}
	\subsection{Advertisments through Google AdWords}
		\begin{description}
			\item[Requirement ID] 43
			\item[Priority] Want to have
			\item[Description] There should be space for advertisements from Google AdWords
		\end{description}
%\section{Performance Requirements}
\section{Database Requirements}
	All data will be saved in the database. The database must always remain consistent. There will be many concurrent request 
	for the database. A good database structure will be needed. 
\section{Design Constraints}
	\begin{enumerate}
		\item The design must be object-oriented
		\item Design diagrams must be made using UML
	\end{enumerate}
\section{Software system attributes}
	\subsection{Reliability}
		Errors in the code will be divided into 2 groups: \textit{small errors} and \textit{fundamental errors}. \\ 
		Fundamental errors will be corrected within 48 hours after the detection of this error. Small errors (details) will be corrected within a week after the detection of these errors. 
%	\subsection{Availability}
	\subsection{Security}
		\begin{itemize}
			\item When a user want to recover his or her password, the site will generate a new password and send it to the user's e-mail.
			\item Communication of sensitive data should be encrypted (e.g. using https)
			\item Users should only be allowed to try to log in a limited number of times
			\item When a guests registers for an account, he or she has to fill in a CAPTCHA
		\end{itemize}
%	\subsection{Maintainability}
%	\subsection{Portability}
%\section{Other Requirements}
\section{Summary}
	\begin{tabular}{|c|c|c|}	
		\hline \textbf{Must have} & \textbf{Want to have} & \textbf{Nice to have} \\ 
		\hline  Create new account & Search a user & View recommended auctions \\ 
		\hline  Log in & View active auctions & Send a personal message \\ 
		\hline  Search for an auction & View placed auctions & View personal messages \\
		\hline  View auctions of a category & Follow an auction & Delete a personal message \\ 
		\hline  Log out & View salespal & Add seller to favourites \\ 
		\hline  Place an auction & Top up salespal & View favourite sellers \\ 
		\hline  Bid on an auction & View buyer's assistant &  Delete a favourite seller\\ 
		\hline  Modify account information & Comment on an auction & Extra site for banned users \\ 
		\hline  Retrieve password & Change currency & Advanced e-mail notification \\ 
		\hline  Remove a user & View auctions with a tag &  \\ 
		\hline  Remove an auction & Cookies &  \\ 
		\hline  Change language & View followed auctions &  \\ 
		\hline  CAPTCHA & View transaction &  \\ 
		\hline  Limited login attempts & Pay transaction  &  \\ 
		\hline  Encrypted password & Rate transaction  &  \\
		\hline  Browser compatibility & Retract bid  &  \\
		\hline  Secure connection & Basic e-mail notification &  \\
		\hline  & Confirm account &  \\
		\hline  & Request new confirmation mail &  \\
		\hline
	\end{tabular} 

%\chapter{Additional Materials}

\begin{thebibliography}{9}
\bibitem{Project} Project description \\
	\url{http://tinf2.vub.ac.be/~dvermeir/courses/software_engineering/projects/2009-2010/index.html}
\bibitem{English} English auction \\ 
	\url{http://en.wikipedia.org/wiki/English_auction}
\bibitem{silent} Silent auction \\
	\url{http://en.wikipedia.org/wiki/Silent_auction#Secondary_types_of_auction}
\end{thebibliography}

%\end{projdoc}
\end{document}