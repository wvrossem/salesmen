%Type document, past een aantal dingen aan. Kort = article, lang = report
\documentclass[a4paper, 12pt]{article}
% Om afkappingstekens op de juiste plaats te zetten
\usepackage[english]{babel}
%Graphicx is voor mooie afbeeldingen, ctable is voor tabellen, url is om links te maken naar internet.
\usepackage{graphicx, ctable, url}
%hyperref is om links binnen uw documenten te leggen, bijvoorbeeld van de bibliografie
\usepackage{hyperref}

%Dit is om source listings te gebruiken, voor meer details lees ftp://ftp.tex.ac.uk/tex-archive/macros/latex/contrib/listings/listings.pdf (gezien je dit later ook vak zal gebruiken)
\usepackage{listings} % source listings
\lstloadlanguages{java}
\lstset{keywordstyle=\ttfamily\bfseries}
\lstset{flexiblecolumns=true}
\lstset{commentstyle=\ttfamily\itshape}

%begint het eigenlijke document
\begin{document}

%geeft titel een naam
\title{Project Management Plan}
%Naam van de schrijver
\author{Patrick Provinciael}
%Date kan ook gebruikt worden als Version History
\date{	29 Octobre, 2009: First draft created \\
	31 Octobre, 2009: Finished the Template}
\maketitle

\section{Overview}

\subsection{Scope}

\section{Something Else}

\end{document}