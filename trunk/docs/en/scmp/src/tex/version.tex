\chapter{Version and Release}
\section{Milestones}
The concept {\it milestone} is used to indicate a goal which is wanted to be achieved within a specified time.
Parallel with the milestones are the releases, see 3.2 Release, and the branches, see 3.3 Branches.
\section{Release}
\subsection{Releases}
A {\it release} is in fact a runnable version of the project which includes all demands, according to the appropriate milestone.
For each milestone, a release is built, this release is also a candidat to be the final working version, if no further releases are made.
In this project there will be 2 to 3 milestones, hence also 2 or 3 releases will be made.
The requirements are playing an important role in the release, especially the ordering of the requirements.
The following milestones are set:
\begin{description}
\item[Milestone 1:]
The first milestone wants the software to include all must-have requirements, in order to be able to build a first release, which can be considered a possible solution for the assignment.
It has to be noted that this first version is not at all a final version, because it contains only the most important features, no extras.
\item[Milestone 2:]
In this stage all (or at least, most of) the want-to-have requirements have been implemented in the software.
The corresponding release will be one which contains important extensions, so that the software has a much better usability.
\item[Milestone 3:]
This should be the final stage, in which the nice-to-have requirements are added to the software. This version has a very good stability and usability.
\end{description}
\subsection{Indexing}
As said above, for each milestone, a release is created. These releases get specific indexes.
\begin{description}
\item[Release 0.1]
This release corresponds with the first milestone, thus containing the must-haves.
\item[Release 0.2]
The second milestone, add of want-to-haves
\item[Release 1.0]
The final, stable version in which the nice-to-haves are included.
\end{description}
\section{Branches}
A {\it branch} is a directory with the same composition of the trunk.
At the moment all must-haves are implemented in the source code in the trunk, the latter is copied to a new branch.
This branch will be named release0\_1.
No further implementations will be made to this release, only fixing bugs will be done on this version.
In the same way the second release will get its own branch, release0\_2, and so on.

