\documentclass{report}

\author{Jorne Laton, Sina Khakbaz Heshmati}
\title{Software Configuration Management Plan}

\begin{document}

\chapter{Configuration Management Planning}
\section{Abbreviations}
In this document a number of abbreviations will be used, which are listed here:
\begin{description}
\item[SCMP]
Software Configuration Management Plan
\item[SCM]
Software Configuration Management
\item[CM]
Configuration Manager
\item[CMA]
Configuration Manager Assistant
\end{description}
\section{Configuration Tree}
This is the structure which will be used to store all source files, reports, webpages and timesheets during the project.
Every team member must be capable of choosing the right directory to commit new files.
If this would not be the case, the CM or the CMA can be contacted to decide where certain new files should be added.
If necessary, this structure can be changed to store a new type of file, but only members of the SCM are authorized to do so, other members are obliged to contact a member of the SCM to request such changes.
For further explanation, see Chapter 2, Change Management.
\subsection{trunk}
This directory contains three important directories.
\begin{description}
\item[docs]
This is where all documentation about the project is stored.
\item[utils]
Utilities can be found here.
To give an example, Timetrack is a utility which is able to convert .xml-format files to other formats, such as .txt and .html.
This utility is used to convert the timesheets, written in .xml, to .html, so that these can be placed on the website in a readable layout.
\item[salesmen]
The project itself: all source files, website files and libraries must be stored here.
\end{description}
\subsection{branches}
You can consider the project development as a timeline, in which the trunk is the main branch, but for each stable release, a new branch is added to the timeline.
More information in Chapter 3, Version and Release.
\subsection{tags}
These are in fact 'snapshots' of the source code at a particular moment in time.
When a stable version is achieved, it can be tagged, see it as a backup.
In this way, when things go wrong implementing new features in a later stadium, not everything is lost, as the latest stable version will still be available.
\subsection{timesheets}
Each team member is responsible for his own timesheet, herefor everyone has to be able to write .xml-files, as the timesheets are to be committed in this format.
The deadline for committing timesheets is set each Monday, at 00h00, thus {\it before} Monday.
To put these timesheets on the website, they are converted into html-format, using Timetrack, see Utilities.
An example of this directory structure is:\\
timesheets/2009/week43/\\
In this directory the timesheets of week 43 of all team members are stored, but only in .xml-format.
\subsection{minutes}
Of each meeting a report is kept by the secretary, this report is made in tex-format.
This is saved in the appropriate directory, for each meeting a new directory is made.
An example of this directory structure is:\\
minutes/2009/10/20/\\
Very self-explanatory: this directory contains all necessary formats (.tex, .txt and .pdf) of the report of the meeting that took place on 20 October 2009.
This date is also included in the filename of the report.
\subsection{wiki}
As it suggests, this is where wikis are stored.
Those wikis can be manuals, agendas or other important documents for team members.
They can easily be edited by every team member, be it via SVN or via Google Code itself.
For example, a Requirements wiki was created to add ideas for requirements.
\section{Subversion}
Subversion has been chosen to regulate checkouts, commits and updates of project files.
The arguments for this choice are the following:
\begin{description}
\item[Flexibility]
Misplaced adds or files with a badly chosen name, can be changed without problems, which is not the case in, for example, CVS, an older version based server system.
\item[Revisions]
In SVN, commits are saved as revisions.
In this way, all changes made to files are monitored very precisely, because each commit causes an increment of the revision index, which is then given to the new version of the file.
Hence, it is possible to compare revisions.
\item[Centralized]
SVN is created for use with centralized servers, which is ideal for project groups in which everyone knows and can contact everyone.
No team member hosts the main directory, this is done by the central server.
The opposite, {\it decentralized}, is used in (mostly open source, like Linux) environments with a lot of programmers, who do not know each other and who work independent.
In a decentralized system the main software is hosted by the manager of the project.
\end{description}
\section{Management Control}
Every team member can checkout and update his working copy of the main software.
To send back changes made to a file of the working copy, one has to commit this file.
When no large changes are made, this should work without problems.
The SCM strongly recommends to work incrementally on files, in other words, committing large changes on different places in a file is not encouraged.
This is to avoid conflicts when trying to merge files edited by different programmers. More about this in the next section, Merging.
If one, for example, wants to implement a new feature, which will drastically change the source code, he has to contact the SCM. The latter will consider this request, but more about this in Chapter 2, Change Management.
\section{Merging}
Imagine two members editing the same file, then committing it to the server at the same time.
This will not be a problem, considering they both worked incrementally on the file, thus not committing a large change at once.
Merging is what regulates this, by recognizing the changes of both sides it can create a new file containing both versions.
It has to be kept in mind that this tool cannot perform miracles.
If two members have worked on the exact same paragraph of a document and then commit, a conflict will occur.
Merging will not provide a solution, but last one who committed will be asked to clear out this conflict by comparing his version with the conflicting version.
\section{Nightly Build}
In the beginning this will not be dealt with, only once the source code is ready to be compiled and run.
From then on, it will be a daily task to have runnable version of the code, which can be built at the end of the day.
A day cannot be closed if the code is not compiled without errors, otherwise the next day would start with a non-runnable version which needs to be debugged first.

\section{Website}

\chapter{Change Management}
{\it Issues} is the keyword in the Change Management, requesting changes happens via issues.
An issue consists of two different sorts of variables: the status and the label(s).

\section{Issue Status Values}
\subsection{Open}
\begin{description}
\item[New]
An issue at this state has not yet received an initial review, it is pending, until the SCM puts it in another state.
\item[Accepted]
At this state, the issue has been acknowledged by the SCM, its need is then confirmed.
\item[Started]
Working on this issue has begun.
\end{description}
\subsection{Closed}
\begin{description}
\item[Fixed]
A developer has made the required changes a source code file or a document.
At this status it is up to the QA to check whether these changes really handled the problem.
\item[Verified]
The QA has checked the fix and has confirmed that it is working.
\item[Invalid]
This issue is rejected by the SCM, it will not be implemented (yet).
\item[Duplicate]
This issue report is a copy of or very similar to an already existing issue, therefor it is rejected.
\item[WontFix]
It has been decided not to take action on this issue.
\item[Done]
The requested task is completed.
\end{description}

\section{Issue Labels}

\subsection{Type}
\begin{description}
\item[Defect]
Reporting a bug in the software.
\item[Enhancement]
Proposing a specific enhancement on the software.
\item[Task]
Work item that does not change the source code or documents.
\item[Review]
Request for a source code review.
\item[Other]
An issue of a type that does not correspond to any of the above listed, will be given type-other.
\end{description}
\subsection{Priority}
\begin{description}
\item[Critical]
Must be resolved in the specified milestone.
\item[High]
Want to be resolved in the specified milestone.
\item[Medium]
Normal priority.
\item[Low]
Does not necessarily need to be implemented in the current milestone, can slip in to later milestone.
\end{description}
\subsection{OpSys}
\begin{description}
\item[All]
Affects all operating systems (Windows, Linux and Mac OS X).
\item[Windows, Linux or OSX]
Affects the respective operating system users.
\item[Firefox, Internet Explorer, Safari]
Affects the respective browser users.
\end{description}
\subsection{Milestone}
\begin{description}
\item[Release0.1]
Must-have requirements are implemented.
\item[Release0.2]
Want-to-have requirements are implemented.
\item[Release1.0]
All essential functionality implemented, tested and working. This ought to be a very stable version.
\end{description}
\subsection{Component}
\begin{description}
\item[UI]
Relates to program User Interface.
\item[Logic]
Relates to application logic.
\item[Persistence]
Relates to data storage components.
\item[Utilities]
Utility and installation scripts.
\item[Docs]
Relates to end-user documentation.
\end{description}
\subsection{Others}
\begin{description}
\item[Security]
Security risk to users.
\item[Performance]
Program performance.
\item[Usability]
Affects program usability.
\item[Maintainability]
Hinders future changes.
\end{description}

\chapter{Version and Release}
\section{Milestones}
\subsection{Roadmap}
\subsection{Branches}
\section{Releases}
\subsection{Indexing}

\chapter{System Build}
\section{Apache Ant}
\subsection{Compile}
\subsection{Javadoc}
\subsection{JUnit}
\subsection{JAr}
\subsection{Dist}
\subsection{Deploy}
\subsection{Run}
\section{Build names convention}

\end{document}
