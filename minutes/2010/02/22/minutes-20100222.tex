\documentclass[a4paper, 12pt]{article}
\usepackage[english]{babel}
\usepackage{graphicx, ctable, url}
\usepackage{hyperref}
\usepackage{eurosym}
\usepackage{listings} % source listings
\lstloadlanguages{java}
\lstset{keywordstyle=\ttfamily\bfseries}
\lstset{flexiblecolumns=true}
\lstset{commentstyle=\ttfamily\itshape}

	\begin{document}

\title{Minutes Meeting 16}
\author{Jonathan Jeurissen}
\date{\today}

\maketitle	
	\section{Meeting Information}
		\textbf{Date:} Friday, February 22, 2009\\
		\textbf{Time:} 18H00--19H00\\
		\textbf{Place:} E.010\\
		\subsection{Attendees}
Following members were present:
			\begin{itemize}
				\item Nick De Cooman
				\item Jonathan Jeurissen
				\item Bart Maes
				\item Jorne Laton
				\item Patrick Provinciael
				\item Wouter Van Rossem
				\item Sina Khakbaz Heshmati
			\end{itemize}
%Following members were absent:
%			\begin{itemize}
			 
%			\end{itemize}
			
		\subsection{Revision History}
			\begin{tabular}{c | l | l }
				\textbf{Rev.} & \textbf{Date} & \textbf{Description} \\
				\hline
				1.0 & February 28, 2009 & First draft created \\
				1.1 & March 1, 2009 & Revised by QA Assistant \\
				%1.2 & December 21, 2009 & Ready to be uploaded \\
			\end{tabular}		

	\section{Purpose of this meeting}
This meeting has been arranged because there were a few complaints about Patrick and Patrick's attitude towards the project and it's members. In this meeting those arguments have been discussed by the whole team.
	\section{Arguments}
			\begin{itemize}
				\item Implementation:\\
\textbf{Offense:} Friday February 19,2010 it looked like Patrick has not implemented anything yet. \\
\textbf{Defense:} Patrick has already done implementation, but has not committed it yet (because his code was not tested yet), which let to believe Patrick did not implement anything yet.

				\item Environment not set-up completely:\\
\textbf{Offense:} In the timesheet of week 51, Patrick claims to have set-up the environment. Quote: "Prepared for the Implementation phase. Including: Ensuring that environment is set-up. Read and prepared the task assigned to me." At February 19th, Patrick was not finished with setting up JBoss and Seam. This became clear in the workshop.
\textbf{Defense:} Patrick had installed Eclipse, with all it's plugins. This was what he meant by setting up the environment. It was a misunderstanding. Patrick could easily install JBoss, but he was waiting for the verdict whether JBoss would be installed on the server or locally (N.B. On the server and locally was the answer to this question, as already explained by Sina). Anyway, he encountered a problem: his laptop was unable to run JBoss and Seam due to a shortage of memory.

				\item No progression, no drive:\\
\textbf{Offense:} According to Nick, there was no drive in Patrick for this project. This statement was not shared by everyone. He didn't come to the workshop, because it didn't seem important enough.
\textbf{Defense:} Patrick has put a lot of work in the project, but sometimes he spends time on the wrong subjects instead of those of his deadlines. Patrick also knows from himself that he has been too lazy sometimes in the past, but he will work on this. However, Patrick says that his drive for this course was strong enough, but that it decreases by these critics.

				\item No test framework:\\
\textbf{Offense:} There is no test framework yet, while the deadline was in December.
\textbf{Defense:} There is a test framework, namely JUnit, but it seemed insufficient for Seam. Seam has its own test framework, but Patrick knew too late that he should have used TestNG instead of JUnit. He still needed time (because of the exams) to study Seam and TestNG, that's why the usable test framework is not finished yet. Wednesday, February 24th, he will give a quick tutorial on how to use TestNG. The framework will be finished by the end of February.

				\item STD not finished yet:\\
\textbf{Offense:} The STD is only 9 pages and only contains basic information. It's not fished yet.
\textbf{Defense:} In the previous meeting this has already been discussed. The STD will be updated in the future, but Patrick wanted to focus on the implementation instead of working on the STD. In his opinion, the Implementation (which needs 2 phases) is more important than finishing the STD.

				\item Attitude:\\
\textbf{Offense:} Sometime Patrick has a negative (or even aggressive) attitude towards the other team members. Arguing is sometimes a problem, as Patrick gets easily infuriated (with yelling in a meeting as a result). Patrick has a strong opinion and sometimes does not like the help or thoughts from others. When he sticks to his opinion (which is not always the one with the best arguments), this can work contra-productive.
\textbf{Defense:} He is aware of this problem and does its best to keep himself under control when arguing. This effort is visible, but still insufficient for now. Also, this critic is a little bit old and there have not been real problems since the last time that there was some yelling in the meetings, as he worked on that. Nevertheless, Patrick also has to be to be a contributor. 
He is aware of his strong opinion and he will try to appreciate the help from others. We're all a team here.

				\item Conventions:\\
\textbf{Offense:} The conventions are not yet complete and are not specific enough.
\textbf{Defense:} The conventions were still general because Patrick didn't want to overload the team with conventions. They will be edited soon and will be put online on the according Wiki page. 

				\item Critics towards the manager:\\
\textbf{Offense:} Nick claims that the line has been crossed too much and too many time and he sent an e-mail with arguments to Patrick, without telling him first in person. This e-mail was for Patrick an aggressive attack and has put him in a weak position.
\textbf{Defense:} There were several opinions about this subject. Some members agreed that Nick should not have send this e-mail to the whole group, but should have privately discussed this with Patrick. The meeting on the other hand was a good idea, as the team wants to help with these problems. Patrick said he was surprised about all the arguments and noted that one should handle a problem when it occurs and not when �the line has been crossed too much'.

			\end{itemize}
    \section{Arguments against Patrick from Patrick himself}
Patrick had prepared a document with a sort of QA against himself and he discussed this shortly. The items in this document are enumerated because he wanted to improve on these subjects. By the way, this document was meant for the previous meeting, but due to a train delay, he could not be there. 
		\begin{itemize}
			\item Communication: \\
			Patrick did too many things without clearly communicating with the others.
			\item Deadlines: \\
			Unimportant deadlines are also important.
			\item Laziness: \\
			Patrick claims to be lazy sometimes.
			\item Timesheets: \\
			His timesheets are wrong, but not because he puts too much on them but often because he didn't put enough references on them. Patrick did give some examples about this. However, sometimes the results of his many hours were not there. Also, occasionally there are a few references missing in his timesheets. 
			\item A few other items, which were already discussed above
		\end{itemize}

	\section{N.B.}
This meeting should also be a warning to the other team members. This project should be taken seriously and there is not much time to lose. This is why this problem has been discussed, in the line of progression. The purpose is that the team gets stronger out of it.
		\section{Agenda}
A complete list of Agenda items is available on \cite{agendaitems}.\\
	
	%\begin{thebibliography}{99}
	%	\bibitem{site1}
	%	\href{http://wilma.vub.ac.be/~se2\_{}0910/dev/events/2009/vub-se/index.html}{http://wilma.vub.ac.be/~se2\_{}0910/dev/events/2009/vub-se/index.html}
		
		
	%	\bibitem{agendaitems}
	%	\href{http://code.google.com/p/salesmen/wiki/NextMeetingTopics}{http://code.google.com/p/salesmen/wiki/NextMeetingTopics}

		
	%\end{thebibliography}	
		
\end{document}
