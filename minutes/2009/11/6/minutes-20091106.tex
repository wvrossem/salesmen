\documentclass[a4paper, 12pt]{article}
\usepackage[english]{babel}
\usepackage{graphicx, ctable, url}
\usepackage{hyperref}
\usepackage{listings} % source listings
\lstloadlanguages{java}
\lstset{keywordstyle=\ttfamily\bfseries}
\lstset{flexiblecolumns=true}
\lstset{commentstyle=\ttfamily\itshape}

	\begin{document}

\title{Minutes Meeting 6}
\author{Jonathan Jeurissen}
\date{\today}

\maketitle	
	\section{Meeting Information}
		\textbf{Date:} Friday, November 6, 2009\\
		\textbf{Time:} 12H00--13H00\\
		\textbf{Place:} CSB Seminaryroom 1\\
		\subsection{Attendees}
Following members were present:
			\begin{itemize}
				\item Nick De Cooman
				\item Jonathan Jeurissen
				\item Sina Khakbaz Heshmati
				\item Jorne Laton
				\item Bart Maes
				\item Wouter Van Rossem
			\end{itemize}
Following members were absent:
			\begin{itemize}
				\item Patrick Provinciael
			\end{itemize}
			
		\subsection{Revision History}
			\begin{tabular}{c | l | l }
				\textbf{Rev.} & \textbf{Date} & \textbf{Description} \\
				\hline
				1.0 & November 6, 2009 & First draft created \\
				1.1 & November 8, 2009 & Ready to be uploaded \\
			\end{tabular}		

	\section{Deadlines}
The SRS deadline will be changed: the first draft needs to be available at Monday, November 9. The deadline of the final version (of the first programming cycle) will be set at Friday, November 13.\\ \\ 
The other deadlines will be changed accordingly.
The design has to be finished Friday, November 27 and the implementation plan needs to be online at Friday, December 4, 2009. At the end of the first program cycle, there will be a unit-testing framework, which will have to be finished on January 8, 2010. All these changes will be summarized in the next version of the SPMP.\\

	\section{Requirements}
		\subsection{Reliability}
Errors in the code will be divided into 2 groups: \textit{small errors} and \textit{fundamental errors}. \\ Fundamental errors will be corrected within 48 hours after the detection of this error. Small errors (details) will be corrected within a week after the detection of these errors. 
		\subsection{Issues}
For every requirement an issue will be made on the Google Groups website\cite{site1}. In this manner, the progress of this specific requirement can be followed. \\ There will also be made issues for errors that have been detected. These issues will be created by the one that discovered the error.  The Implementation Manager will decide who needs to fix this error (mostly, but not necessarily, the author of this code).
		\subsection{Layout}
The layout of Chapter 3 of the current version of the SRS will be changed. In the next version, there will be a clear distinction between the different requirements. The requirements will be numbered with a code, corresponding to their type.
		\subsection{Want to have}
A new requirement that will be added to ``Want to have'' requirements is: \textit{Cookies}. These cookies will allow users to save their data during session.
	\section{\LaTeX{} template}
There is a need for a new Salesmen logo, with less colors to obtain a more professional look. Also, another implementation of the revision history in the \LaTeX{} documents can be useful.

	\section{SCMP}
The SCMP has been thoroughly discussed and explained to the team members. It will be finished (and reviewed) by Monday, November 9, as planned. It is important that the team members understand and learn how to use the SCMP. This is applicable to all documents in general written about this project. The purpose of these documents is that team members use them for a better understanding of the project.

	\section{Next Meeting}

		\textbf{Date:} Tuesday, November 10, 2009\\
		\textbf{Time:} 15H00--17H00\\
		\textbf{Place:} D2.10\\
	
		\subsection{Agenda}
A complete list of Agenda items is available on \cite{site3}.
	
	\begin{thebibliography}{99}
	
		\bibitem{site1}
		\href{http://code.google.com/p/salesmen/issues/list}{http://code.google.com/p/salesmen/issues/list}
		
		\bibitem{site3}
		\href{http://code.google.com/p/salesmen/wiki/NextMeetingTopics}{http://code.google.com/p/salesmen/wiki/NextMeetingTopics}
		

		

		
	\end{thebibliography}	
		
\end{document}
