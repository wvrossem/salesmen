\documentclass[a4paper, 12pt]{article}
\usepackage[english]{babel}
\usepackage{graphicx, ctable, url}
\usepackage{hyperref}
\usepackage{listings} % source listings
\lstloadlanguages{java}
\lstset{keywordstyle=\ttfamily\bfseries}
\lstset{flexiblecolumns=true}
\lstset{commentstyle=\ttfamily\itshape}

	\begin{document}

\title{Minutes Meeting 8}
\author{Jonathan Jeurissen}
\date{\today}

\maketitle	
	\section{Meeting Information}
		\textbf{Date:} Tuesday, November 17, 2009\\
		\textbf{Time:} 15H00--17H00\\
		\textbf{Place:} D2.06\\
		\subsection{Attendees}
Following members were present:
			\begin{itemize}
				\item Nick De Cooman
				\item Jonathan Jeurissen
				\item Sina Khakbaz Heshmati
				\item Bart Maes
				\item Jorne Laton
				\item Patrick Provinciael
				\item Wouter Van Rossem
			\end{itemize}
There were no members absent.
			
		\subsection{Revision History}
			\begin{tabular}{c | l | l }
				\textbf{Rev.} & \textbf{Date} & \textbf{Description} \\
				\hline
				1.0 & November 18, 2009 & First draft created \\
				1.1 & November 19, 2009 & Revised by the QA Team \\
				1.2 & November 19, 2009 & Revision on HTTPS \\
			\end{tabular}		

	\section{Logo}
A few logo's have been shown, but none of the current logo's were elected as the future Salesmen logo. There has been some discussion about the ``men'' in Salesmen, because it might be female unfriendly. Perhaps a future logo that will suppress this undesirability will be made, but there has not been a verdict concerning this issue yet. \\
New logo's can still be made by the team members and should be uploaded before Sunday, November 22 at 23H00.
	\section{SRS}
There has not been any feedback from the client yet, so there cannot be a discussion about what the client desires just yet.
		\subsection{Adaptations}
In a last review of the SRS, a few requirements were still missing. These omissions will be shortly discussed in this paragraph and will be added to the next version of the SRS.
			\begin{itemize}
				\item Browser Compatibility\\
The Salesmen website will be compatible with the most significant modern browsers. It will be compatible with Firefox 3.0(and higher), Internet Explorer 7(and higher) and Safari 3.2(and higher). 
%The support for one browser will be added as a \textit{Must have} requirement, while supporting several browsers will be a \textit{Want to have} requirement.
				\item Returning after Login \\
A user will return to the page he was visiting after he logged on to the website. 
				\item Limiting page changes \\
When surfing on Salesmen, a user will stay in the same page for as long as possible. e.g. He will not go to a new page when bidding on an object. This would be a nice feature to improve ease of use of the website.
				\item Disallowing users under 18 year \\
There has been decided that adding this kind of requirements (consisting of rules that apply according to law) is the responsibility of the client. The SRS is a contract. If the client wants to add these sort of requirements, he will need to tell this. At this moment, it is not clear if this kind of requirements need to be added to the SRS. 
			\end{itemize}

	\section{Design}
		\subsection{Web Design}
		\begin{itemize}
		\item Fixed width \\
The Salesmen website will be made with a fixed width to support most of the monitors currently used. The width will thus be set at 950 pixels to fit into monitors with a width of 1024 pixels. Even nettop computers use this width.
		\item Advertising \\
The free space left and right of the Salesmen can be used for advertising to earn extra money.
		\item HTTPS \\
		As a possible measure to improve website security, the HTTPS protocol
		may be used. However, one must take into account that this will
		increase the load on the webserver.
		\item Layout \\
There will be a logo, a login field, a top navigation bar (My Salesmen, Buy, Sell,etc.), a search field, sub-navigation (hierarchy), a bid button, information about the seller, arrows for browsing through different items, etc. \\
Soon, there will be a prototype available of this design on the Google Groups website.
		\end{itemize}

		\subsection{Implementation Design}
The initial class diagram for the project has been shown to the team members and shortly discussed.

	\section{Conference on Tuesday, December 15}
In four weeks, there will be a conference about the progress of the  project. As this is an important conference, there will be held a separate meeting  on Friday, November 27, dedicated to the topics and presentation(s) on the conference. As a preparation for this meeting there will be a brainstorm on Tueday, November 24.

	\section{Quality Assurance Presentation}
There has been a presentation about the different aspects of Quality Assuarance. The topics about this presentation will be summarized in the SQAP.
	\section{Tools and technologies}
At the moment, the different tools (e.g. J2EE, Apache Tomcat, etc.) that will be used are not well enough known by the team members. In order to fix this issue, the Configuration Management team will provide the team members with sufficient information about these tools. This will be done by a presentation in the next meeting, as with a Wiki page that will be made concerning the available tools. One should not rely on the expertise of an individual for using a certain tool. This is why the Wiki page will be made at least two days before the next meeting, so the team members can study and understand the terms that will be used to in the presentation. Also, they will be able to discuss the choice of certain tools.

	\section{Risk Management}
Most risks that were documented in this project are not valid risks. Thus, the risks of this project will be reviewed.
		\begin{itemize}
			\item What if the Google Server does not work anymore or shuts down for a certain period? \\
\textit{Solution:} \\
Backups need to be made from all files on the SVN Server. A script will be made by Jorne to download all files once a day.
			\item What if there is a lack of knowledge concerning techniques or tools? What if certain tools are difficult or completely new to team members?\\
\textit{Solution:} \\
Tutorials will be made as Wiki pages. Ultimately presentations can be given if a Wiki page is not a good option for a complete explanation. However, a Wiki page is the first step to explaining new tools.

			\item How can the compatibility with different browsers be assured? \\
\textit{Solution:} \\
There will have to be extensive testing after changes to the project. There is however a risk that the compatibility for a certain browser will be lost. To minimize compatibility problems, standards will be used.

		\end{itemize}

	\section{Next Meeting}

		\textbf{Date:} Tuesday, November 24, 2009\\
		\textbf{Time:} 15H00--17H00\\
		\textbf{Place:} D2.10\\
	
		\subsection{Agenda}
A complete list of Agenda items is available on \cite{site1}.\\
	
	\begin{thebibliography}{99}
	
		
		\bibitem{site1}
		\href{http://code.google.com/p/salesmen/wiki/NextMeetingTopics}{http://code.google.com/p/salesmen/wiki/NextMeetingTopics}
		

		

		
	\end{thebibliography}	
		
\end{document}
