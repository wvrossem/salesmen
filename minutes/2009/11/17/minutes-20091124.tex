\documentclass[a4paper, 12pt]{article}
\usepackage[english]{babel}
\usepackage{graphicx, ctable, url}
\usepackage{hyperref}
\usepackage{listings} % source listings
\lstloadlanguages{java}
\lstset{keywordstyle=\ttfamily\bfseries}
\lstset{flexiblecolumns=true}
\lstset{commentstyle=\ttfamily\itshape}

	\begin{document}

\title{Minutes Meeting 9}
\author{Jonathan Jeurissen}
\date{\today}

\maketitle	
	\section{Meeting Information}
		\textbf{Date:} Tuesday, November 24, 2009\\
		\textbf{Time:} 15H00--17H00\\
		\textbf{Place:} D2.06\\
		\subsection{Attendees}
Following members were present:
			\begin{itemize}
				\item Nick De Cooman
				\item Jonathan Jeurissen
				\item Sina Khakbaz Heshmati
				\item Bart Maes
				\item Jorne Laton
				\item Patrick Provinciael
				\item Wouter Van Rossem
			\end{itemize}
There were no members absent.
			
		\subsection{Revision History}
			\begin{tabular}{c | l | l }
				\textbf{Rev.} & \textbf{Date} & \textbf{Description} \\
				\hline
				1.0 & November 25, 2009 & First draft created \\
				1.1 & November 26, 2009 & Revised by the QA Manager \\

			\end{tabular}		

	\section{Tools}
Sina explained the different tools and terms which will be used in this project in a short presentation. Terms like ``Multitiered application'', ``Servlet'', ``Application server'',``JVM'', ``API'', ��Web Container'',``EJB'' and so on, where explained. A comprehensive list of the terms and tools can be found at the Wiki page \textit{SalesmenLiterature} \cite{site1}.
		\subsection{Editor}
Most team members have already had experience with using Eclipse, for code editing. This is the main reason why Eclipse will be used as the Code editor of this project.
		\subsection{Application server}
The choice of the application server has been thoroughly discussed by the team. Tomcat, Glassfish and JBoss where a few of the candidates, each with there own advantages. Some applications are more compatible with certain editors, than others. JBoss is very known and very stable. It provides an official plugin for Eclipse, which was the most important argument for choosing JBoss. As an extra, JBoss supports a hibernate function. 
		\subsection{Database}
MySQL has been chosen for the Salesmen's database because it's free (open), good, stable, fast and commonly used. Also, a few members already have experience with MySQL.
		\subsection{Programming Language}
Java EE will be used for the programming language of this project. Scale and Groovy are two other options, there has been decided that this project will program in the the standard way.
		\subsection{Google Web Toolkit}
Google Web Toolkit (GWT) translates Java to HTML and can be very handy for this project. There has been a lot of discussing and explaining about GWT and whether or not to use it, but the bottom line is, that nobody really knows the functions of GWT very well. This is why this choice has been delayed till the next meeting.
%Javascript gets information asynchronously from a server. At the moment it is not clear if GWT support search engine optimisation, using Ajax.
%Actual content of the website has to be returned by the server, not by Javascript. If this is possible by GWT, this would be great.
%These are a few reasons why there is a lack of information concerning the use of GWT.
	\section{Logo and template}
		\subsection{Logo}
A few logo's have been proposed. Bart's logo \cite{site4} will be chosen if it will have nicer colors and if it would be two separate logo's: one containing only the word ``Salesmen'' and one containing only the short logo (with the letter S).
		\subsection{template}
The template is almost finished. There will be made a new front page with a more professional look, including the logo of Salesmen and the logo of the VUB. 

	\section{Design}
		\subsection{Web Design}
There is a prototype available of this design SVN Server. \cite{site5}
		\subsection{Implementation Design}
A fairly complete class diagram for Salesmen has been shown and shortly discussed. There were no important changes necessary at the moment. The scheme will be put online by Friday, November 27-

	\section{Brainstorm about Conference I}
Following topics might be discussed at the conference:
		\begin{itemize}
		\item Website prototypes
		\item Logo
		\item Statistics about planning, cost and working hours
		\item Innovative ideas \\
This will be a very important topic for the client. This will show how Salesmen is different from the other projects, so there will be a lot of time spent on this topic. \\
e.g. Several ways to earn money with this project.
e.g. The search assistant of Salesmen.
		\item The look and feel of the website
		\item Choices made in this project (tools etc.)
		\item Summary of the plans
		\item SRS: general description of a few interesting requirements
		\item Configuration Management and SVN
		\item Optimization \\
This is an interesting topic, but it might be that important for the client, so it will be omitted at the conference.
		\end{itemize}
There is a good possibility that there will be duo-presentations or dialogs to make the presentation a bit more special.
\\
This is only a short description of what will be presented at the conference. In the next meeting, all topics will be thoroughly discussed.
	
	\section{Blog}
There will be made a blog on the Google Groups website, in which changes can be publiced. This blog will act as a notebook of the experencies of the team members.
	\section{SPMP}
A Risk table has been added and the current cost of the project has been calculated. At the moment the team members did work 283 hours on the project, which means it equals about 4245� (at 15�/hour). 
The total hours of the team members will be put in statistics. Also, there has been decided to use \textit{Ohloh} to clear up
	\section{SRS}
There will be made a new Wiki page to post new requirements. If these requirements have been discussed and approved in the upcoming meeting, they will be added to the SRS and deleted from this Wiki page. This will make it more clear for the team which requirements will be added.
		\subsection{Adaptations}
			\begin{itemize}
				\item Earning money \\
There will have to be more requirements that clear up how the client can earn money with the site. Requirements such as advertising, Google AdWords, etc. will be added to the new version of the SRS.
				\item Unactivated Users \\
A new type of user will be added: unactivated users. These are users who have just registered on Salesmen, but who did not activate their account yet.
				\item Costs
There has to be requirements that specifies all the costs for placing an object and for selling an object (e.g. 5\% of the selling price)
			\end{itemize}
For a full list, one can look at the difference report of the SRS. \cite{site2}
			
	\section{SCMP}
The SCMP of Salesmen has been compared to that of the other groups. There were a few difference such as:
			\begin{itemize}
				\item Another layout
				\item Configuration items
				\item Locking files \\
The possibility to lock a certain file, to be able to work on it, without anyone changing it.
				\item Control of changes \\
In the other groups there is a difference between small and little changes. Eeach change has it's own consequences and needs to be approved by the Implementation Manager. This will not be used in our project.
				\item Weekly backups instead of daily backups
				\item Status report of the repository (History) \\
There is a good possibility that this change will be used in our project in the future.
			\end{itemize}
In general, the SCMP of Salesmen is more complete than that of the other groups. For a full list, one can look at the difference report of the SCMP. \cite{site3}
			
	
	\section{SQAP}
There tools in Eclipse to ease the use of coding conventions. The Quality Assurance Manager (Patrick) will propose some coding conventions at the meeting on Tuesday, December 1. 

	\section{Next Meeting}

		\textbf{Date:} Friday, November 27, 2009\\
		\textbf{Time:} 12H00--13H00\\
		\textbf{Place:} CSB Seminaryroom 1\\
	
		\subsection{Agenda}
A complete list of Agenda items is available on \cite{site6}.\\
	
	\begin{thebibliography}{99}
		\bibitem{site1}
{http://code.google.com/p/salesmen/wiki/SalesmenLiterature}{http://code.google.com/p/salesmen/wiki/SalesmenLiterature}

		\bibitem{site2}
		\bibitem{site3}
		\bibitem{site4}
{/attic/2009/11/logo/bart-3.png}{}
		\bibitem{site5}
{/attic/2009/11/website-prototype}{} 
		\bibitem{site6}
		\href{http://code.google.com/p/salesmen/wiki/NextMeetingTopics}{http://code.google.com/p/salesmen/wiki/NextMeetingTopics}
		

		

		
	\end{thebibliography}	
		
\end{document}
