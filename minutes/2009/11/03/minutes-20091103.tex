\documentclass[a4paper, 12pt]{article}
\usepackage[english]{babel}
\usepackage{graphicx, ctable, url}
\usepackage{hyperref}
\usepackage{listings} % source listings
\lstloadlanguages{java}
\lstset{keywordstyle=\ttfamily\bfseries}
\lstset{flexiblecolumns=true}
\lstset{commentstyle=\ttfamily\itshape}

	\begin{document}

\title{Minutes Meeting 5}
\author{Jonathan Jeurissen}
\date{\today}

\maketitle	
	\section{Meeting Information}
			\textbf{Date:} Tuesday, November 3, 2009\\
			\textbf{Time:} 15H00--17H30\\
			\textbf{Place:} D2.10\\
		\subsection{Attendees}
Following member were present:
			\begin{itemize}
				\item Nick De Cooman
				\item Jonathan Jeurissen
				\item Sina Khakbaz Heshmati
				\item Jorne Laton
				\item Bart Maes
				\item Patrick Provinciael
				\item Wouter Van Rossem
			\end{itemize}
There were no members absent.
		\subsection{Revision History}
			\begin{tabular}{c | l | l }
				\textbf{Rev.} & \textbf{Date} & \textbf{Description} \\
				\hline
				1.0 & November 3, 2009 & First draft created \\
				1.1 & November 4, 2009 & Ready to be uploaded \\
			\end{tabular}		

	\section{Requirements}
The first thing that needed to be discussed was the naming of the categories. Some members preferred Core, Extensions and Bloats over ``Must have'', ``Want to have'' and ``Nice to have''. There has been decided to use the latter, because these were not only the client's own words but they are also the most simple description of the requirements.\\However, there has been a disagreement about whether or not to use the term ``Core". By ``Core'' one means the truly basic functions (the absolute minimum) this project needs to be called an auction site. e.g. There is no need for a two browsing functions, as only one would be enough for a very basic auction site. After a long discussion, there has been decided not to use the term ``Core'', as it's more related to Implementation. Also, introducing ``Core'' caused ambiguity within the team members. For these reasons, this term will not been used in the futher development of the process. A solution has been found by reducing the ``Must have'' requirements to a minimum. Only those functions that are necessary to  deliver a reasonably good auction site will be included in this type of requirements.
\\ \\
The different requirements have been thoroughly discussed in this meeting and have been divided into one of the three chosen subcategories.\\ \textit{Disclaimer:}\\
Following requirements are only a basic, incomplete and non-binding description, as discussed in the meeting on Tuesday, November 3, 2009. For a full description of the requirements, one should look at the SRS of this project or for a pre-final version at the Wiki page of the Requirements\cite{site1}.

		\subsection{Must have}
These features will certainly be included in the project.
			\begin{itemize}
				\item User database
				\item Login (secure possibility to identification)
				\item Auctions
				\item Categories
				\item Search function
				\item One language
				\item Support for additional languages (i18n)
				\item Simple search engine
				\item Secure website
			\end{itemize}
	
		\subsection{Want to have}
There is a good probability that these features will be included.
			\begin{itemize}
				\item User ratings
				\item Tags
				\item Second language
				\item User home
				\item Basic e-mail notification
				\item Virtual money account
				\item Support for Paypal
				\item Private Messaging service between users

			\end{itemize}
		
		\subsection{Nice to have}
Under the condition that there is still time left, following features might be included in the project.
			\begin{itemize}
				\item Advanced e-mail notification
				\item Support for different currencies
				\item Currency converter
			\end{itemize}
			
	\section{\LaTeX{} template}
There will be a new template available on the SVN Server on Wednesday, November 4 at 23H00. This template will include a Salesmen Logo. Also, there will be a possibility to use this template within the \LaTeX{} documents by using one command (instead of copy-pasting the code). This template will be used by all other documents written in \LaTeX{}. This includes the documents that already have been published onto the website.

	\section{Risk Analysis}
A list of the current risks will be kept in the SPMP\cite{site2}. These are the new risks discussed at the last meeting:\\
	Current Risks:
		\begin{enumerate}
		\item  What if a ``Must have'' requirement can't be achieved?\\
		Solution:\\
Every team member will have to spent more time until all ``Must have'' requirements are complete. 
		\item  What if a team member does not know enough Java? What if a team member has problems with the techniques used in this project?\\
		Solution:\\
Team members will have to study -- or at least look up -- the syntax of Java and Javascript, as well as principles and methods which we will use in this project. New methods will also be discussed in the meetings.
		\end{enumerate}
	\section{SCMP}
A first draft of the SCMP has been released on the SVN Server, but due to problems with \LaTeX{}, this draft has not been published on the website yet. However, it will be published as soon as possible.

	\section{Project Planning}
Every team member will write down what he has done in the previous two weeks. He will also say what he will do in the coming two weeks and his plans for the coming year. In this manner there will be a clear view over the work done and the work that needs to be done. These meetings allow teams to discuss their work, focusing especially on areas of overlap and integration. This is often called SCRUM.
		
	\section{Next Meeting}

		\textbf{Date:} Friday, November 6, 2009\\
		\textbf{Time:} 12H00--13H00\\
		\textbf{Place:} CSB Seminaryroom 1\\
	
		\subsection{Agenda}
A complete list of Agenda items is available on \cite{site3}.
	
	\begin{thebibliography}{99}
	
		\bibitem{site1}
		\href{http://code.google.com/p/salesmen/wiki/Requirements}{http://code.google.com/p/salesmen/wiki/Requirements}
		
		\bibitem{site2}
		\href{http://wilma.vub.ac.be/~se\_{}20910/docs/spmp/index.html}{http://wilma.vub.ac.be/~se\_{}20910/docs/spmp/index.html}
		
		\bibitem{site3}
		\href{http://code.google.com/p/salesmen/wiki/NextMeetingTopics}{http://code.google.com/p/salesmen/wiki/NextMeetingTopics}
		

		

		
	\end{thebibliography}	
		
\end{document}
