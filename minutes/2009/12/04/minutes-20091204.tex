\documentclass[a4paper, 12pt]{article}
\usepackage[english]{babel}
\usepackage{graphicx, ctable, url}
\usepackage{hyperref}
\usepackage{eurosym}
\usepackage{listings} % source listings
\lstloadlanguages{java}
\lstset{keywordstyle=\ttfamily\bfseries}
\lstset{flexiblecolumns=true}
\lstset{commentstyle=\ttfamily\itshape}

	\begin{document}

\title{Minutes Meeting 11}
\author{Jonathan Jeurissen}
\date{\today}

\maketitle	
	\section{Meeting Information}
		\textbf{Date:} Friday, December 04, 2009\\
		\textbf{Time:} 12H00--13H00\\
		\textbf{Place:} CSB Seminaryroom 2\\
		\subsection{Attendees}
Following members were present:
			\begin{itemize}
				\item Nick De Cooman
				\item Jonathan Jeurissen
				\item Sina Khakbaz Heshmati
				\item Bart Maes
				\item Jorne Laton
				\item Patrick Provinciael
				\item Wouter Van Rossem
			\end{itemize}
There were no members absent.
			
		\subsection{Revision History}
			\begin{tabular}{c | l | l }
				\textbf{Rev.} & \textbf{Date} & \textbf{Description} \\
				\hline
				1.0 & December 5, 2009 & First draft created \\
				1.1 & December 6, 2009 & Revised by QA Manager \\

			\end{tabular}		

	\section{Logo}
The logo is now finished and has a more professional look. It can be found on the SVN at \cite{site1}.
	\section{Next Meetings}
			\begin{tabular}{l | l }
				\textbf{Date Meeting} & \textbf{Description} \\
				\hline
				Tuesday, December 8, 2009 & Preparation conference \\
				Friday, December 11, 2009 & Weekly meeting \\
				Tuesday, December 8, 2009 & Conference I \\
				Friday, December 11, 2009 & Weekly meeting \\
				Friday, January 29, 2010 & Last meeting of the first iteration \\
			\end{tabular}	
	\section{Analysis statistics}
		The several statistics of the workings hours for every team member have been walked through. Clearly Sina has done too much work in comparison to the other team members. All the other team members work approximately the same per week(around 6 hours/week). \\
		At the first week, there was a high peak for all team members to start up the project. For every team member, there was a peak in the week of the deadline of his report(SRS,SDD, SPMP,etc.). For the other weeks, every team member worked approximately 6 hours/week (except for Sina). \\ Averaged over all weeks and all members, every member works approximately 7 hours/week. 
		
	%\section{Prepare Conference 1}

	\section{SQAP}
A 2nd repository will be added on the SVN, in which only the QA team will commit code. The purpose of this repository is to clearly show which code is Quality Assured. If there would not be a 2nd repository, than Quality Assured code would have to be set as a comment in the code, which will not be very efficient when the code should be changed. It would not be clear which code is Quality Assured and which is not

I would like a 2nd to the code repository, this repository commit just me (bart and in some cases).  The idea comes from a number of assistants that I have talked about our project (especially a certain "problems" which I have encountered include the QA). This idea is the only thing you will be affected immediately. The other ideas that you have influence on them is coming later.
	\section{Implementation}
JBoss Seam will be used as a framework. This has as advantage that the implementation will be separate from the logic. So xHTML will be separate from the logic. It will also use very clean code. A programmer will only have to make logic and for example not use the Java Search Pages. \\
The classes of the SDD will represent the data. There will also be a class that will consist of auctions that can be performed on other classes. \\
Everyone has to install JBoss 5 and JDK 6. A Wiki page will be created that will explain how this should be installed and how examples of JBoss Seam can be ran.\cite{site2}. \\
Sina will provide a list of tasks that can be done by the team members by next meeting.

	\section{Requirements: support several languages}
This project will support different languages. These languages will be supported, by using an extra String table, which will be used to import Strings that have to be displayed on the website. The description of the items itself can be translated by using Google Translate. It's important to understand that the first version of Salesmen will only be in English, but it will be possible to add a new language, just using a different String table.
	
	\section{Next Meeting}

		\textbf{Date:} Tuesday, December 4, 2009\\
		\textbf{Time:} 15H00--17H00\\
		\textbf{Place:} D2.10\\
	
		\subsection{Agenda}
A complete list of Agenda items is available on \cite{site6}.\\
	
	\begin{thebibliography}{99}
		\bibitem{site1}
		\href{http://code.google.com/p/salesmen/source/browse/attic/2009/11/logo/final/bart-3.png}{http://code.google.com/p/salesmen/source/browse/attic/2009/11/logo/final/bart-3.png}
		
		\bibitem{site2}	\href{http://code.google.com/p/salesmen/wiki/HowtoRunJBossSeamExamples}{http://code.google.com/p/salesmen/wiki/HowtoRunJBossSeamExamples}
		
		
		\bibitem{site6}
		\href{http://code.google.com/p/salesmen/wiki/NextMeetingTopics}{http://code.google.com/p/salesmen/wiki/NextMeetingTopics}

		
	\end{thebibliography}	
		
\end{document}
