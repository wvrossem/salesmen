\documentclass[a4paper, 12pt]{article}
\usepackage[english]{babel}
\usepackage{graphicx, ctable, url}
\usepackage{hyperref}
\usepackage{eurosym}
\usepackage{listings} % source listings
\lstloadlanguages{java}
\lstset{keywordstyle=\ttfamily\bfseries}
\lstset{flexiblecolumns=true}
\lstset{commentstyle=\ttfamily\itshape}

	\begin{document}

\title{Minutes Meeting 10}
\author{Jonathan Jeurissen}
\date{\today}

\maketitle	
	\section{Meeting Information}
		\textbf{Date:} Tuesday, December 01, 2009\\
		\textbf{Time:} 15H00--17H00\\
		\textbf{Place:} D2.10\\
		\subsection{Attendees}
Following members were present:
			\begin{itemize}
				\item Nick De Cooman
				\item Jonathan Jeurissen
				\item Sina Khakbaz Heshmati
				\item Bart Maes
				\item Jorne Laton
				\item Patrick Provinciael
				\item Wouter Van Rossem
			\end{itemize}
There were no members absent.
			
		\subsection{Revision History}
			\begin{tabular}{c | l | l }
				\textbf{Rev.} & \textbf{Date} & \textbf{Description} \\
				\hline
				1.0 & December 2, 2009 & First draft created \\
				1.1 & December 3, 2009 & Revised by QA Manager \\

			\end{tabular}		

	\section{Logo and template}
		\subsection{Logo}
A few variants to the Bart's original Salesmen logo have been proposed. However the original logo (with the orange S) has been elected as the logo for Salesmen.
		\subsection{template}
The \LaTeX{} template is will finished by Friday, December 4.
	\section{To do list}
The to do list of the previous week has been discussed. The report of the presentation about Quality Assurance still needs to be made. Jorne's and Wouter's report about the differences in the SCMP and SRS are completed.

	\section{MySQL vs. PostgreSQL}
Instead of MySQL, Salesmen is going to use PostgreSQL as database. PostgreSQL has been chosen because it has more features than MySQL and because it's generally more complete. It has features like for instance Foreign Key(this is a key in a certain table that links to another table), which can be very handy in this project. Normally, no SQL-statements should be written, because EJB3 will generate this code.
Much more information about a comparison between MySQL and PostgreSQL has been discussed by mail under the topic MySQL vs PostgreSQL \cite{site1}.

	\section{SDD}
The SDD has greatly improved, but there are still a few details that will need to be adapted (e.g. how transactions will be made, additional methods will need to be included). In other words, the current SDD is not yet complete. \\
When the requirements will be more specified, the SDD will change accordingly. At this moment, the \textit{use cases} of the SRS are included at the SDD. This means that whenever the requirements in the SRS will change, the SDD will have to be updated as well. This is not really a problem, as the SDD will then have to be checked for possible changes anyway.\\
The design has to be as complete as possible: it has to include most \textit{Must have} and \textit{Want to have} requirements. The Implementation will start with a few \textit{Must have} requirements and then proceed with the \textit{Want to have's}. This does not mean that the SDD should not yet include these \textit{Want to have} requirements.\\

	\section{Implementation}
Next week, the team will start will the implementation of Salesmen. This will be possible because the conventions will be discussed in this meeting and will be decided by Friday. The design doesn't have to fully completed to start the implementation. A few things can already be done. The Implementation Manager (Sina) will give the team members implementation jobs soon.(e.g. Issues must be created for every requirement)\\
Users need to be implemented as \textit{beans}. This means they consist of getters and setters. This is not yet the case in the current version of the SDD. This is one of the reason why the Implementation Manager and the Design Manager will need to discuss the SDD thoroughly to make an efficient implementation of the code possible. For instance, beans can be generated by Eclipse, if the SDD defines them correctly.
		
	\section{Competitor Analysis}
There has been chosen to use Dropbox for internal documents, which will not be published on the website. All members of the team will create their own Dropbox account and a Dropbox folder will then be shared. The Implementation Manager (Jorne) will take care of the management of Dropbox.
	
	\section{Preparation of Conference I}
There are several subjects that will be discussed on the conference. A short description about the subjects will be given below:
		\begin{description}
			\item \textbf{General description} \\
A general description of the Salesmen project and the team will be given, the statistics and the cost of the project will be shown. Also, the planning, deadlines and what's been done so far shall be discussed.
			\item \textbf{Innovation and requirements} \\
It will be explained how Salesmen will make the life of buyers, sellers and the owner of the site easier. Requirements meant especially for this job will be explained. The fact that there is a possibility to add tags to an auction and that objects can be recommended using tags from existing auctions. Also, the possibilities Salesmen creates for the owner to earn money will be thoroughly presented. 
			\item \textbf{Configuration and tools} \\
Following items will be clarified: the timesheets of Salesmen, Google Code (and it's possibility for SVN ),the possibility to browse the SVN through a web browser and finally code reviews.
			\item \textbf{Quality Assurance} \\
Especially the differences with the other groups will be shown. Possibly a few key words that are important in the Quality Assurance of Salesmen.
			\item \textbf{Pitch} \\
A summary of the strengths of Salesmen, showing why Salesmen is the right choice.
 	\end{description}
 	
	\section{Risk Analysis}
A list of the current risks will be kept in the SPMP. These are the new risks discussed at the last meeting:\\
		\begin{enumerate}
		\item  A lack of knowledge about the tools\\
		Solution:\\
Knowledge about Java: It seems that the team members are still in touch with the syntax and implementation of Java code, so a presentation or Wiki page will not be necessary. \\
Knowledge about Eclipse: A short tutorial on how to use Eclipse will be given on a Wiki page of the Google Code website. \\
Knowledge about other tools: Wiki pages will be supplied (e.g. the SalesmenLiterature Wiki page)
		\item Examinations: The team will have to learn for the examinations.\\
		Solution:\\
There will be a code freeze on Thursday, December 30, 2009.
		\end{enumerate}
	\section{Highlights of November}
A document concerning the Highlights of November is available on the website. This can still be reviewed (or adapted) by the team members untill tomorrow Wednesday, December 2 and will then be e-mailed to the client.
	\section{SCRUM}
A template for the SCRUM is available on the SVN. The template consists of 3 parts:
		\begin{enumerate}
			\item What a team member has been doing for the last 2 weeks.
			\item What a team member will do for the coming 2 weeks.
			\item Which problems a team member will possible encounter.
		\end{enumerate}
Every 2 weeks, the SCRUMs will be discussed in the meeting.

	\section{SRS}
A few requirements have been posted on the Wiki page and will thus be discussed and afterwards added to the SRS.
		\subsection{New requirements}
			\begin{itemize}
				\item Transaction costs \textit{(Must have)} \\
A certain percentage of every transaction will go to the owner of Salesmen.
				\item Changing transaction costs \textit{(Nice to have)}\\
There should be an ability for the client to set the transaction costs to his preference.
				\item Space for Advertising \textit{(Must have)}\\
Every page of the Salesmen website has to include space for advertising (e.g. Google Adds).
				\item Platinum members \\
Users can upgrade their free account to a Platinum account for a certain price. This special membership will hide the ads on the website and can possibly imply a reduction on the costs for placing an item on the Salesmen website. There has been an idea to add a Trust Symbol for Platinum users. It's not clear whether or not this Trust Symbol will be used on the Salesmen website. Only the items of Platinum members can be selected for the \textit{Hot deals} on the front page of the Salesmen website.
				\item Clean URL's \\
Salesmen will be using clean URL's (e.g. ``http://www.salesmen.com/auction/1523-VW-Golf3-1996" instead of ``http://www.salesmen.com/auctiondetail.jsp?action-id=1523"), so the URL's become readable and the addresses have a clean impression.
				\item Hot deals \\
Specials items from Platinum members, which have a high page view ranking, will be displayed on the front page of the Salesmen website. Administrators should be able manage these hot deals. 
			\end{itemize}
The SRS will be updated before Sunday, December 6 at 23H00.

	\section{Conventions}
A few examples of conventions have been proposed. (e.g. Will we use varname, Varname, varName or var\_{}name in the code). The team members declared their opinion about the different choices. The coding conventions will be put on a Wiki page by Tuesday, December 8.  


	\section{Next Meeting}

		\textbf{Date:} Friday, December 4, 2009\\
		\textbf{Time:} 12H00--13H00\\
		\textbf{Place:} CSB Seminaryroom 3\\
	
		\subsection{Agenda}
A complete list of Agenda items is available on \cite{site6}.\\
	
	\begin{thebibliography}{99}
		\bibitem{site1}
		\href{http://groups.google.com/group/salesmen/browse\_{}thread/thread/beed4cc3c1953887}{http://groups.google.com/group/salesmen/browse\_{}thread/thread/beed4cc3c1953887}
		
		\bibitem{site4}	\href{http://code.google.com/p/salesmen/source/browse/svn/attic/2009/11/logo/bart-3.png}{http://code.google.com/p/salesmen/source/browse/svn/attic/2009/11/logo/bart-3.png}
		
		\bibitem{site5}	\href{http://code.google.com/p/salesmen/source/browse/svn/attic/2009/11/website-prototype}{http://code.google.com/p/salesmen/source/browse/svn/attic/2009/11/website-prototype}
		
		\bibitem{site6}
		\href{http://code.google.com/p/salesmen/wiki/NextMeetingTopics}{http://code.google.com/p/salesmen/wiki/NextMeetingTopics}

		
	\end{thebibliography}	
		
\end{document}
