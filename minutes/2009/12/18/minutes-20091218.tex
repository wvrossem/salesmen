\documentclass[a4paper, 12pt]{article}
\usepackage[english]{babel}
\usepackage{graphicx, ctable, url}
\usepackage{hyperref}
\usepackage{eurosym}
\usepackage{listings} % source listings
\lstloadlanguages{java}
\lstset{keywordstyle=\ttfamily\bfseries}
\lstset{flexiblecolumns=true}
\lstset{commentstyle=\ttfamily\itshape}

	\begin{document}

\title{Minutes Meeting 15}
\author{Jonathan Jeurissen}
\date{\today}

\maketitle	
	\section{Meeting Information}
		\textbf{Date:} Friday, December 18, 2009\\
		\textbf{Time:} 12H00--13H00\\
		\textbf{Place:} CSB Seminaryroom 2\\
		\subsection{Attendees}
Following members were present:
			\begin{itemize}
				\item Nick De Cooman
				\item Jonathan Jeurissen
				\item Bart Maes
				\item Jorne Laton
				\item Patrick Provinciael
				\item Wouter Van Rossem
			\end{itemize}
Following members were absent:
			\begin{itemize}
			 \item Sina Khakbaz Heshmati
			\end{itemize}
			
		\subsection{Revision History}
			\begin{tabular}{c | l | l }
				\textbf{Rev.} & \textbf{Date} & \textbf{Description} \\
				\hline
				1.0 & December 20, 2009 & First draft created \\
				%1.1 & December 20, 2009 & Revised by QA Manager \\
				%1.2 & December 21, 2009 & Ready to be uploaded \\
			\end{tabular}		

	\section{Evaluation conference 15/12}
	The conference went very well. Overall it was rather smooth (as planned) and the content was rightly chosen. The Salesmen presentation really focused on which differences they made over other buying/selling sites. This aspect has been brought very clearly. There were a few remarks about the presentation though:
		\begin{itemize}
			\item The part of the configuration was not natural. It was a little bit forced. A duo presentation is good idea, but it was not smooth due to too short sentences of Jorne and a time line that was too strictly planned.
			\item The duo presentation abouts the requirements and the design on the other hand went very well. This shows that with a little more exercise, duo presentation can really work.
			\item The conference might be a little bit too much like businessmen that were talking, but in a way, this was the purpose. This project is called Salesmen, and that what Salesmen do.
			\item The original presentation was much smoother, but this smoothness has faded a little bit because of the questions in between. Next time, there will be asked to ask questions at the end.
		\end{itemize}
		
In Group 1, the part of the comparison between the different tools was very good. Salesmen could actually learn a bit from that.\\
The slides of the conference have been put online on the Wilma server. \cite{site1} A nice interactive gadget to view the slides directly on the website, has been added by Sina. Also the documents concerning the conference were put online on the same website.

	\section{Timesheets}
The references in the timesheets of the last two weeks will be updated, as they were scrapped from the website until after the conference. The time spent at the conference will also be put in the timesheet of every team member, as this in fact a part of the project. \\
Timesheets will still need to be committed every week, even if there is nothing on them.
	\section{SCRUM}
All SCRUMS have been sent to the Project Manager, except the one from Sina. They will be bundled together and summarized. Afterwards this summary will be put online.
 
	\section{\LaTeX{} Template}
There will be a feature to add the revision history in the .opt file. A new VUB logo will also be added.
	\section{QA Documents}
All documents (except for the Minutes) will be updated accordingly to the template. This will be done by Patrick. All documents will now be consistent and it will also be clear that they are from Salesmen. 
	\section{Implementation}
		\subsection{JBoss and Seam on server}
		JBoss and Seam start really slow on an average local computer. This is why there has been proposed to use them on the server. It would be handy to install JBoss and Seam on the Wilma server and commit's can be to this server in order to test the code.This involves a clear division between the code that's been worked on and the working Salesmen code. An interaction between the different parts (auctions, users, etc.) will be necessary when implementing. This has  implications on the unit testing of the code. Patrick will look in to this problem, as this is the task of a QA Manager. The installation of JBoss and Seam (and possibly PostgreSQL) will done by Bart en Nick. This will be done by Monday, December 21. It would be best if Sina then tried to connect JBoss with the database. 
		\subsection{Workloads}
There will be done some implementing between between December 18 and December 31. Sina will need to give jobs to the team members in order for this to happen. Jonathan has been excused to implement during this period, because of his lower study points for this course.
		\subsection{Method}
As the Implementation Manager was absent in this meetings, the different tasks and the method has not been discussed yet.	Sina will need to pick up his task of Implementation Manager more seriously. He does a lot of work on other things, like maintaining the website, but the Implementation is now really more urgent. Nick will send Sina an e-mail to clear this up.
	\section{Next Meeting}
		A date for the next meeting has not yet been set.
%
%		\textbf{Date:} Friday, January 29, 2009\\
%		\textbf{Time:} 15H00--17H00\\
%		\textbf{Place:} D2.10\\
%	
		\subsection{Agenda}
A complete list of Agenda items is available on \cite{agendaitems}.\\
	
	\begin{thebibliography}{99}
		\bibitem{site1}
		\href{http://wilma.vub.ac.be/~se2\_{}0910/dev/events/2009/vub-se/index.html}{http://wilma.vub.ac.be/~se2\_{}0910/dev/events/2009/vub-se/index.html}
		
		
		\bibitem{agendaitems}
		\href{http://code.google.com/p/salesmen/wiki/NextMeetingTopics}{http://code.google.com/p/salesmen/wiki/NextMeetingTopics}

		
	\end{thebibliography}	
		
\end{document}
