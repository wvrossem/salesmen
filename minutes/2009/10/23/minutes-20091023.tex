\documentclass[a4paper, 12pt]{article}
\usepackage{ctable}
\usepackage{url}
\usepackage{hyperref}

\title{Software Engineering Group 2 (2009-2010) \\Meeting Report}
\author{Jonathan Jeurissen}
\date{ Tuesday, October 20, 2009, 12H00--13H00 in CSB Seminaryroom 1}



\begin{document}

	\maketitle
	
	\begin{tabular}{l l}
		{\large \textbf{Attendees}} \\
		Nick De Cooman & Present \\
		Jonathan Jeurissen & Present \\
		Sina Khakbaz Heshmati & Present \\
		Jorne Laton & Present \\
		Bart Maes & Present \\
		Patrick Provinciael & Present \\
		Wouter Van Rossem & Present \\
		\\
	\end{tabular}	
	
	
	
	
	\section{Website}
	
	\subsection{\LaTeX{} template}
		In terms of consistency, there is a need for a template file to make the \\LaTex{} documents. To be able to use the same style in every \\LaTeX{} document, a template specifically designed for Salesmen, will be created by Patrick. This template will incorporate a new Salesmen logo.
	\subsection{Timesheet template}
A timesheet template is now available on the website\cite{site1}. How this code should be used is explained on the same webpage. Basically, every member of the team needs to add his own .XML code in a so called ``TimeTrack'' file to the correct place on the server. The collection of the .XML codes can be automatically put together and a .pdf can be extracted.
	\subsection{SVN}
The SVN consists of 5 parts: Trunk (main development branch), Timesheets (summary of time spent by each member), Minutes (reports of meetings), Branches and Tags (snapshots of stable version). Only the first 3 need to be checked out, when a team member wants to update or add files. More information of the SVN can be found in the Wiki of the Salesmen Google Code website\cite{site2}. It is absolutely necessary that everyone uses and understands SVN.
A tutorial on how to use SVN (on Linux and on Windows) will be prepared, so everybody can manage their own files, without Sina's personal help. To automate the uploading even more, there will be worked on a script that converts .tex files to .pdf and .txt.


	\section{SPMP}
The current version of the SPMP has been presented to the team. Especially the deadlines and the summary of tasks(Organizational structure) were thoroughly debated. 
	 \subsection{Deadlines}
There were only a few minor changes in the deadlines:
	\begin{itemize}
		\item Minutes:\\
The deadline for the Minutes will be changed to 3 days after the meeting. The first version of the Minutes should be finished 2 days after the meeting and will then be inspected by the Quality Assurance Team. The third day after the meetings, the final version of the Minutes need to be online on the website.
		\item Timesheet:\\
Everyones personal timesheet should be filled out every Sunday by 00H00 (instead of the previous 20H00). This decision has been made because some members would like to work between 20H00 and 00H00 on a Sunday evening.
		\item SRS:\\
The first version of the SRS needs to be released on Monday, November 2, 2009. The second version (to be seen as a pre-final version) needs to be online on Monday, November 9, 2009. As this project uses the spiral model, there will possibly be made new versions of the SRS after the first program cycle.
	\end{itemize}	
	\subsection{Proces Model}
In the SPMP a schematic of the future work flows needs to be made. At this moment, there was only an example available of the spiral model. A schematic, specifically designed for Salesmen, will be made.
	\subsection{Organizational structure}
There was a bit of incertitude about who should talk with the client and also about the specific task of the Design Manager concerning the website. These uncertainties should be clarified in the next version of the SPMP.
	\section{Documenting methods}
There was a small disagreement about whether or not methods that were not chosen in this project should be documented. An example of this problem can be found below. There has been decided to focus on the chosen methods. If there is a need to mention alternatives, they should be shortly noted or referred in a comparative way.
	\begin{description}
		\item [Example of the problem] As the spiral model has been chosen, should the disadvantages of the waterfall model be explained ?
		\item [Solution] The waterfall model has been considered as an alternative, but it was discarded in favor of the spiral method.
	\end{description}
	
	\section{Replying e-mails}
From now on, everybody needs use \href{mailto:se2@tinf.vub.ac.be}{\nolinkurl{se2@tinf.vub.ac.be}} for all mails (also when they are meant for a specific team member. However, there has been an agreement on subject syntax for those kinds of e-mails that are sent to specific team members of the group. In this way, it will be clearer which mails are relevant for each team member and it will still be possible to read all mails, if desired.
For a specific e-mail, one will then have to use:\\
``[SpecificName(s)] Subject''\\
When sending mails to the whole group, the names don't need to be specified.



	\section{Risk Analysis}
A list of the current risks will be kept on the Wiki page. These are the new risks discussed at the last meeting:\\
	Current Risks:
		\begin{enumerate}
		\item What if the client can't make an appointment before the deadline? \\
		Solution:\\
The client should be contacted as soon as possible. Update: The client has been contacted on Friday, October 23, 2009 and an appointment has been made on Monday, October 26, 2009.
		\item What if the the wrong questions have been asked to the client?\\
		Solution:\\
There should be at least two appointments with the client before the first version of the SRS will be available. Also, a full list of the questions\cite{site3} needs to be available at the website, so every team member can add his own questions.
		\item What if the requirements are misunderstood or misinterpreted?\\
		Solution:\\
The interview with the client will be recorded and will be made available to all team members (if this is allowed). This interview will be relistened by the Quality Assurance manager to reduce misinterpretations. In order to avoid ambiguity, there will be a presentation of the requirements in another appointment with the client.
		\end{enumerate}



	\section{Next Meeting}
	
	Next meeting will be on Fiday 30, 2009 at 12H00, in CSB Seminaryroom 1. 
	
	
	\subsection{Agenda}
	As there is a deadline for the Minutes Report, the complete list of Agenda items cannot be included in this document, but are available on \cite{site4}.
	%\begin{itemize}

			
		%\item 
		
	%\end{itemize}
	
	
	\begin{thebibliography}{99}
		
		\bibitem{site1}
		\href{http://code.google.com/p/salesmen/wiki/HowtoSubmitTimesheet}{http://code.google.com/p/salesmen/wiki/HowtoSubmitTimesheet}

		\bibitem{site2}
		\href{http://code.google.com/p/salesmen/wiki/HowtoSvnCommandline}{http://code.google.com/p/salesmen/wiki/HowtoSvnCommandline}

		\bibitem{site3}
		\href{http://code.google.com/p/salesmen/wiki/RequirementsInterview}{http://code.google.com/p/salesmen/wiki/RequirementsInterview}
		
		\bibitem{site4}
		\href{http://code.google.com/p/salesmen/wiki/NextMeetingTopics}{http://code.google.com/p/salesmen/wiki/NextMeetingTopics}
		
	\end{thebibliography}	
	
	\begin{center}
	 	Last revised on \today.
	\end{center}
	
	
\end{document}
