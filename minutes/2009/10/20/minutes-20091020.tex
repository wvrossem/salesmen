\documentclass[a4paper, 12pt]{article}
\usepackage{ctable}
\usepackage{url}

\title{Software Engineering Group 2 (2009-2010) \\Meeting Report}
\author{Jonathan Jeurissen}
\date{ Tuesday, October 20, 2009, 16H45--18H00 in D2.10}



\begin{document}

	\maketitle
	
	\begin{tabular}{l l}
		{\large \textbf{Attendees}} \\
		Nick De Cooman & Present \\
		Jonathan Jeurissen & Present \\
		Sina Khakbaz Heshmati & Present \\
		Jorne Laton & Present \\
		Bart Maes & Present \\
		Patrick Provinciael & Present \\
		Wouter Van Rossem & Present \\
		\\
	\end{tabular}	
	
	
	\section{Job Assignments}
There has been decided that "Backup" is not an appropriate term for the assistant of every function in this project. Also there is a new function "Project Secretary". The Secretary will e.g. write the reports of the meetings.
These are the definitive roles that we have assigned on the meeting on Tuesday, October 20, 2009 : 
	
	\begin{tabular}{l l l}
		\\
		\FL Role & Effective & Assistant
		\ML Project Leader & Nick & Patrick
		\NN Project Secretary & Jonathan & Wouter
		\NN Configuration Manager & Jorne & Sina
		\NN Quality Assurance Manager & Patrick & Bart
		\NN Requirements Manager & Wouter & Jonathan
		\NN Design Manager & Bart & Nick
		\NN Implementation Manager & Sina & Jorne
		\NN Webmaster & Sina & Wouter
		\LL
		\\
	
	\end{tabular} 
	
	\section{Website}
	
	\subsection{Lay-out}
	\begin{itemize}
		\item Grouping of items \\
	As an extra feature, documents on the website will be grouped per meeting. At the moment there are only the different revisions available, as one big list. This feature will improve 		usability and readabilty. 
		\item Tutorial \\
		A tutorial on writing/editing/updating documents and integrating these documents on the server, will be written by the Configuration Management Team.
	\end{itemize}
	\subsection{Issue Management}
	From now on, issues will be posted on the website. These issues require solutions to achieve a milestone. In these issues, comments can be given by all the team members. In this way, we can achieve our milestones in an efficient manner.
	\subsection{Responsibilities}
	A Wiki page will be made, to clear up the responsibilities of everyone. This page shall be managed by the Project Leader. He will clearly specify the tasks of every member. These specifications will be revised by the team members themselves.

	\section{Agreements}
	\subsection{Deadlines}
	A few deadlines have been set:
	
	\begin{itemize}

		\item Timesheets\\
		Everyone has to update his .XML code every Sunday before 20H00.
		\item SPMP\\
		The first version of the SPMP (Software Project Manager Plan) will be made by Monday, October 26, 2009 at 20H00.
		\item SRS Questions\\
		A Wiki page will be made to post possible questions to ask our client. Update: This Wiki page has already been created on Wednesday, October 21,2009 by Wouter\cite{site1}. 		The first version of the SRS will be finished by Wednesday, October 28,2009. 	
	\end{itemize}
	

	\subsection{Timesheets}
	The timesheet shall be filled out be every member in .XML coding. A template based .XML coding has been chosen to realise a straightforward solution. There's no need for an extra program or function on our website. The Project Leader will put all the .XML codes together every Sunday. This complete .XML document, can be easily converted to the appropriate necessary file types. For extra information, one could look at \cite{site2}.
	
	\subsection{Language}
	The language used in the code (and comments) will be English for the internationality of the code. To be consistent in every aspect, all documents will be written in English.

	\section{SPMP}
	The spiral model with a few cycles will be used for this project, because it has a high amount of risk analysis, combined with the advantage that the software will be produced in an early stage of the project life cycle. This will improve the motivation of the team members. Both waterfall method and extreme programming have been considered als alternatives, but they were discarded in favor of the spiral method.

	\section{Tools}
	\begin{itemize}
	\item {Google Code} \\
	There has been chosen for Google Code because it has some handy features:
	\begin{enumerate}
 		\item Revision control system (Subversion, for extra information, see below).
		\item Issue tracker
		\item Wiki (is sometimes handy e.g. for meeting agenda).
		\item File download server (for our software and document releases).
	\end{enumerate}
	
	\item Subversion\\
	This is a centralized version control system chosen for the following reasons:
		\begin{itemize}
	\item It is easy to refactor the source code structure, while preserving files' history.
	\item The whole repository has a single revision that is incremented after each commit.

		\end{itemize}

	\item Eclipse\\
	Eclipse is a software development environment comprising an IDE and a plug-in system to extend it. It is used to develop applications in Java. \cite{site3}
	There has been decided to work with Eclipse because it's very userfriendly interface and it's popularity. Also it's know by most of the team members.
	\item \LaTeX{}\\
	 \LaTeX{} has been chosen to document this project because it's internationally known and commonly used. Also, a few members were interested in learning this 			language. Wouter, Sina and Jonathan have taken a course on Tuesday, October 20,2009 to study \LaTeX{}, to be able to write their reports.

	\end{itemize}

	\section{Risk Analysis}
	New risks will be discussed at every meeting. A list of the current risks will be kept on the Wiki page.\\
	Current Risks:
		\begin{enumerate}
		\item What if Google goes down for a certain period?\\
		Solution:\\
		Backups of our files and documentation will be performed by Jorne and possibly assisted by Sina.
		\item What if somebody becomes ill?\\
		Solution:\\
		This risk has already been considered in the beginning of this project. This is why there is an Assistant(formerly called Backup) for every Management position. If a leader or manager gets ill, the assistant of that function should be able to fully understand his function and replace that leader or manager, for a certain period of time.
		\item What if somebody doesn't spend enough time on his job?\\
		Solution:\\
		At first this will be discussed at the next meeting and a suggested action will be given. Thereafter a warning will be given. Possibly there will be further actions. If it's absolutely 			necessary, a member can be dismissed. In this case, the solution to the risk stated above applies.
		\end{enumerate}



	\section{Next Meeting}
	
	Next meeting will be on Fiday 23, 2009 at 12H00, in E1. 
	
	
	\subsection{Agenda}
	
	\begin{itemize}

			
		\item Risk Analysis
		
		\item Latex template for documents
		
		\item SPMP
		
		\item Summary of tasks
		
		\item SVN
		
		\item Requirements
		
		\item Timesheets
		
		\item Documenting methodes that were not chosen
		
		\item Replying e-mails
		
	\end{itemize}
	
	
	\begin{thebibliography}{99}
		
		\bibitem{site1}
		http://code.google.com/p/salesmen/wiki/RequirementsInterview
		\bibitem{site2}
		http://code.google.com/p/salesmen/wiki/RiskAnalysis
		\bibitem{site3}
		http://en.wikipedia.org/wiki/Eclipse\_{}java
		
	\end{thebibliography}	
	
	\begin{center}
	 	Last revised on \today.
	\end{center}
	
	
\end{document}