\documentclass[a4paper, 12pt]{article}
\usepackage{ctable}
\usepackage{url}
\usepackage{hyperref}

\title{Software Engineering Group 2 (2009-2010) \\Meeting Report}
\author{Jonathan Jeurissen}
\date{ Friday, October 30, 2009, 12H00--13H00 in CSB Seminaryroom 1}



\begin{document}

	\maketitle
	
	\begin{tabular}{l l}
		{\large \textbf{Attendees}} \\
		Nick De Cooman & Present \\
		Jonathan Jeurissen & Present \\
		Sina Khakbaz Heshmati & Present \\
		Jorne Laton & Present \\
		Bart Maes & Present \\
		Patrick Provinciael & Present \\
		Wouter Van Rossem & Present \\
		\\
	\end{tabular}	
	
	
	
	
	\section{Website}
	
	\subsection{\LaTeX{} template}
		This template is already in it's last stage and will be available on the website on Saturday October 31, 2009.
	\subsection{General}
The website is almost finished and will now get a lower priority than the development of the project itself.
	\section{SPMP}
The current version of the SPMP is reasonably good: there were only a few minor changes necessary. Chapter 5 of the SPMP is not finished yet, but this is normal at this state of the project.
	\subsection{Features}
A risk table will be made for the SPMP by Monday. Also, a Grantt Chart, including the different subproblems, will be made.
	\subsection{Agreements and Deadlines}
The deadlines written in the SPMP should be taken into account. The ``Minutes'' of Friday, October 23, was not published on time on the website. However, it was available on the SVN Server on Saturday, October 24 but still needed a review of the Quality Assurance team. As the Quality Assurance Manager was not available that weekend and the Assistant was not informed in time, that deadline was not met. In the future, the Assistant needs to be contacted sooner, either by the Manager or by another team member if the Manager does not respond in a reasonable time.\\
As an extra attempt to obtain the deadlines, they will be published on the website as road maps or milestones.
	\section{SCMP}
The final version of the SCMP will be available on the website on Monday, November 2, 2009 as defined in the SPMP.
	\section{Quality Assurance}
The role of the Quality Assurance Manager in this project was not clear enough. It seemed like the Quality Manager had more work than strictly necessary. This is why this role has been thoroughly discussed in this meeting. There has been decided that it is important to be consistent all over the project and that the role of the Quality Assurance Manager has to be taken seriously. In the future, he will also look for consistency in the code (for example: ambiguous names of variables should be avoided). This is a public project, so names of variables should be chosen carefully. However unnecessary changes in the code are not important. One should always consider the time spent on optimizing the code versus just renaming variables without adding extra functionality or readability. For checking the code, the Quality Assurance Team will create specific tests to improve the code.
	\section{Requirements}
The only requirement the client has specified was that he wanted to earn a lot of money with this auction site. How this site should be made can be specified by the team. Which features will be added, will be decided by the team. Thereafter, they will be presented to the client for approval. The more features and functionality, the better.
The requirements will be divided into 3 classes: ``Must have'', ``Want to have''  and ``Nice to have''. This is only a basic, incomplete and non-binding description of the Requirements, as discussed in the meeting on Friday, October 30, 2009. For a full description of the requirements, one should look at the SRS of this project or for a pre-final version at the Wiki page the Requirements\cite{site1}.

	\subsection{Must have}
	These features will certainly be included in the project.
		\begin{itemize}
			\item Database of the clients
			\item Possibility to identification (login)
			\item Items on the website
			\item Categories
			\item Search function
			\item Different languages (at least 2 languages)
			\item English action
		\end{itemize}
	
	\subsection{Want to have}
	There is a good probability that these features will be included.
		\begin{itemize}
			\item Advanced search possibilities
			\item Different currencies supported
			\item Ratings
			\item Silent auction, Dutch auction, Fixed price 
		\end{itemize}
		
	\subsection{Nice to have}
	Under the condition that there is still time left, following features might be included in the project.
		\begin{itemize}
			\item Concurrency converter
		\end{itemize}
	\subsection{Agenda}
	A complete list of Agenda items is available on \cite{site2}.
	%\begin{itemize}

			
		%\item 
		
	%\end{itemize}
	
	
	\begin{thebibliography}{99}
		\bibitem{site1}
		\href{http://code.google.com/p/salesmen/wiki/Requirements}{http://code.google.com/p/salesmen/wiki/Requirements}
		\bibitem{site2}
		\href{http://code.google.com/p/salesmen/wiki/NextMeetingTopics}{http://code.google.com/p/salesmen/wiki/NextMeetingTopics}

		
	\end{thebibliography}	
	
	\begin{center}
	 	Last revised on \today.
	\end{center}
	
	
\end{document}
